Test case reduction has long been seen as essential to effective automated testing.  However, traditional test case reduction simply reduces the \emph{length} of a test case, but does not attempt to reduce \emph{semantic} complexity.  This paper improves on previous test reduction efforts with algorithms for \emph{normalizing} and \emph{generalizing} test cases.  Rewriting test cases into a \emph{normal form} can reduce semantic complexity and, often, remove steps from an already delta-debugged test case.  Moreover, normalization dramatically reduces the \emph{number} of test cases that a reader must examine, partially addressing the ``fuzzer taming'' problem of discovering all faults in a large set of failing test cases.  Generalization, in contrast, takes a test case and reports what aspects of the test could have been changed while preserving the property that the test fails.  These algorithms rely on the features of a recently introduced domain-specific language, TSTL.  Normalization plus generalization aids understanding of test cases, including tests for TSTL itself and for complex and widely used Python APIs such as the NumPy numeric computation library and the ArcPy GIS scripting package.  Normalization frequently reduces the number of test cases to be examined by \emph{well over an order of magnitude}, and often to just one test case per fault.  Together, ideally, normalization and generalization allow a user to replace reading a large set of test cases that vary in unimportant ways with reading \emph{one annotated test case, summarizing an entire family of similar failures}.
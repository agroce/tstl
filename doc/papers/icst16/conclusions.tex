\section{Conclusions and Future Work}

This paper presents test case normalization and generalization, first
steps towards a goal we suspect can only be approximated: providing
users of automated testing with a \emph{single test case, as short and
  simple as possible, for each underlying fault in the SUT}, where
that test case 1) produces a failure due to the fault and 2) includes
annotations describing the general conditions under which the fault
manifests as failure.  This is the ideal; normalization moves toward
this ideal by rewriting numerous distinct failing test cases into a
smaller set of simpler test cases, and generalization is a
first-effort at an annotation method for describing the core structure
of a failure, separating essential and accidental aspects of a test
case.  Our algorithms rely on the ability of TSTL \cite{NFM15,ISSTA15}, a language for
defining test cases for automated testing, to define simplification
and experiments for generalization in a form amenable to simple
rewriting, which can apply to any SUT and any language.  The approach applies to any test generation
method, including those that produce short tests \cite{FA11,SoftBET}.

Our current implementation is meant to serve as a useful starting
point for further development.  TSTL is available in a working version
\cite{tstl} that supports normalization and generalization, and the
normalization and generalization implementations are designed to allow
the easy addition of further rewrite rules that can move testing even closer to
the goal of ``one test case to rule them all.''

The goal of normalization, not just
reduction, can also be pursued in settings other than API sequence or
string grammar testing.  The difficulties of defining a normal form
for, e.g., JavaScript \cite{jsfunfuzz} or C \cite{CReduce} test cases
are formidable, but even less effective methods might aid debugging and assist fuzzer taming
\cite{PLDI13}.  Simple generalization (e.g., is this numeric constant
essential, can these two statements be swapped?) and a limited form of
fresh value generalization should be easy to apply
even for complex programming language test cases.

\section{Conclusions and Future Work}

This paper presents test case normalization and generalization, first
steps towards a goal we suspect can only be approximated: providing
users of automated testing with a \emph{single test case, as short and
  simple as possible, for each underlying fault in the SUT}, where
that test case 1) produces a failure due to the fault and 2) includes
annotations describing the general conditions under which the fault
manifests as failure.  This is the ideal; normalization moves toward
this ideal by rewriting numerous distinct failing test cases into a
smaller set of simpler test cases, and generalization is a
first-effort at an annotation method for describing the core structure
of a failure, separating essential and accidental aspects of a test
case.  Our algorithms rely on the ability of TSTL, a language for
defining test cases for automated testing, to define simplification
and experiments for generalization in a form amenable to simple
rewriting, which can apply to any SUT and any language.

Our current implementation is meant to serve as a useful starting
point for further development.

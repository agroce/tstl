\section{The TSTL Harness Language}
\label{sec:lang}

The TSTL compiler takes as input a harness template file, and produces
as output a Python class file that implements an interface other tools
(or even users working interactively)
can use to perform testing on the SUT via an SUT-independent interface.

The harness in Figure \ref{fig:MakeFeatureLayer} shows many of the
basic features of TSTL.  The basic structure of a TSTL harness
consists of three parts, usually written in order.  First, harness
code prefixed by an {\tt @} or enclosed in {\tt <@ @>} is treated as
raw Python code, and essentially not interpreted by the TSTL
compiler.  This code is reproduced almost literally in the output
file\footnote{TSTL does have to scan {\tt import}s to re-load modules, and also pre-processes function
  definitions to support
pre- and post-conditions.}.  Second, there is a preamble that almost
always defines a set of \emph{value pools} for use in testing, but
also may include information on logging, correctness properties,
source code locations for code coverage analysis, and other basic
information that applies to the entire harness.  Finally, the bulk of
a TSTL harness (and the only non-optional element) is a set of
\emph{action definitions}.  Actions are the possible steps to be taken in
testing, and define the set of possible tests.

The original version of TSTL \cite{NFM15} required cumbersome use of Python
functions to implement many simple operations, including guards.  Current TSTL extends
the language to make it possible to define very complex test spaces
using only pools and actions, with helper functions only required for
the usual reasons of abstraction and readability.

\subsection{The Essentials of Pools and Actions}

In TSTL, tests usually consist of assignments to value pools and
function calls making use of those values.  These are the most common
forms of actions.  Value pools are meta-variables in the target
language, and support the complete set of types of the underlying
language.  A pool can contain simple types such as integers, or more
complex types such as functions, container classes, file handlers, or
even TSTL testing objects (to support testing TSTL itself).  The
notion of an action in TSTL is similiarly completely general:
\emph{any fragment of code in the host language can be an action}.
Usually, it is most convenient to encapsulate complex actions by
defining functions that perform the desired behavior, and making the
action a call to such a function, but this is not required.  As
Andrews et al. have shown \cite{AndrewsTR}, this pool-and-action approach is
sufficient to express the full generality of unit tests, in any
language\footnote{In fact, TSTL tests are somewhat more general than
  this already very general and expressive form, in that we do not
  disallow loops and conditions in actions.}.

In order to make the core ideas clear,
consider part of the harness shown in Figure \ref{fig:MakeFeatureLayer}, defining how to
generate values used in SQL where clauses.  The following, by itself,
is a valid TSTL harness (albeit one that cannot discover any
faults, since it performs no actions beyond simple integer addition):

{\scriptsize
\begin{code}
pools:
  <val> 2 CONST
\vspace{0.05in}
actions:
\vspace{0.05in}
<val> := <1..10>
<val> = <val> + 1
\end{code}
}

There is only one pool, named {\tt val} (optionally labeled as {\tt CONST} to
indicate that its value does not change unless it appears on the left
hand side of an assignment).  The pool has room to store
two values.  The state of the SUT is defined by the state of all
pools.  Initially, all pools are set to a special value ({\tt None}) indicating the
pool has not been initialized.  For the most part, we can think of the
{\tt val} pool as two Python variables {\tt val0} and {\tt val1}.  Another
way to think about a pool is as a kind of informal ``named type,''
with a limited set of variables that can contain the ``type'' and all
possible action sequences that assign to its pool values serving as
its specification.  In Python it is not necessary to specify the
actual type
of a value pool, though an optional {\tt : type} notation enables TSTL
to perform runtime type-checking and ensure pools never contain
incorrect types.

In this simple example, the only actions are initialization of a {\tt
  val} and incrementing a {\tt val}.  Again, we emphasize that an
action can in general be an arbitrary Python statement.  Actions that
include the {\tt :=} form of assignment (a TSTL, not Python,
operation) initialize pool values.  When {\tt <val>} appears in an
action, that represents all possible pool values with that name: for
our simple example, either {\tt val0} or {\tt val1}.  An integer range
is represented by {\tt <i..j>}, and TSTL expands such ranges to
produce an action with each possible choice. The first line in the
actions section of this harness translates to 20 different possible
actions:

\begin{figure}
\includegraphics[width=\columnwidth]{states}
\caption{Constraints on actions in a test, based on pool states}
\label{fig:poolacts}
\end{figure}

{\scriptsize
\begin{code}
val0 = 1
val0 = 2 ...
val0 = 10
val1 = 1 ...
val1 = 10
\end{code}
}

From the initial state of the system, only these 20 actions are
\emph{enabled}.  \emph{Enabled} actions are those that can be executed in the
current state; the complete set of actions defined by a TSTL harness is always
finite, and the enabled set is always a subset of that finite set.  The first concept that is essential to understanding
TSTL semantics is that at any state of the
system, the only actions that are enabled are those that do not
\emph{use} any non-initialized pool values.  Any appearance of a pool
value is considered a \emph{use}, with the single exception of the left-hand-side of a
{\tt :=} initialization (not normal assignment)\footnote{The definition of
use is the only distinction between {\tt :=} and normal Python
assignment; {\tt :=} is implemented as Python assignment, and appears
as such when test cases are printed.}.  The second concept is that a
value that has been initialized
cannot be initialized (appear on the lhs of {\tt :=}) until after at
least one action
that uses it has been executed.  Figure \ref{fig:poolacts} shows the
consequences of these rules for the simple value assignment harness
above.  The nodes in the graph are labeled with {(\tt state(val0),
  state(val1))}, where state is
either {\tt None} (uninitialized), {\tt Unused} (initialized
but never used) or {\tt Used} (initialized and used at least once).
Starting from the initial state {\tt (None, None)}, a valid test is any path
through the graph. 

Tests that can be produced by this harness include, therefore,
sequences like  {\tt val0 = 3; val0 = val0 + 1; val1 = 4; val1 = val0
  + 2} and {\tt val1 = 10; val0 = 6; val0 = val1 + 1; val0 = 2; val1 =
  15}.  However, {\tt val0 = val0 + 1; val0 = 2} and
{\tt val0 = 1; val1 = 1; val1 = 4} are not valid tests, because they
either use an uninitialized pool value, or re-initialize an unused
pool (a clearly useless action sequence).

\subsection{Guards, Post-Conditions, and Properties}
\label{sec:property}

The example TSTL harness in Figure \ref{fig:MakeFeatureLayer} shows a
few other important core elements of TSTL.  First, choice templates
are not limited to integer ranges, but can include arbitrary items in
a list, e.g, {\tt <fc> := <["d1.shp", "d2.shp", "d3.shp"]>}.  Note
that while in the example these items are (string) constants, they can
be arbitrary expressions to be computed at runtime, or even incomplete
code fragments that are only valid when combined with the rest of the action.
Second, when an action raises an uncaught exception, this is normally
considered a test failure.  Prefixing an action with a set of
exception names in curly braces (e.g., {\tt \{IOError\}}) indicates
that some exceptions are expected, and do not indicate a failure.

More critically, actions can also be prefixed by arbitrary guards, using
the syntax {\tt guard -> action}.  The simple ArcPy harness chooses
field names for SQL by first extracting a list of all fields in some
feature class.  It then allows a field name to be chosen by taking the
name of the first field in the list.  However, since the harness also
allows the list of fields to be stepped through by discarding the
initial element, the name extraction has to be guarded to ensure that
tests won't try to extract names from an empty field list:  {\tt
  len(<fieldlist,1>) >= 1 -> <fieldname> := <fieldlist> [0].name}.
The {\tt <fieldlist,1>} construct, which can also be used outside of a
guard, indicates that this pool value should not be produced using
normal template expansion (instantiated as both {\tt fieldname0} and
{\tt fieldname1}) but rather that it should copy (textually, not a
copy of the object but the same variable use) the comma indexed
appearance of that pool in each expansion (indexing starts from 1).  This makes sure the guard
is over the same pool value that is used in the action.

TSTL also supports post-conditions on actions, in the form {\tt action
  => post-condition}, where the post-condition is checked after the
action is performed.  For example, because some known ArcPy bugs
involve addition of incorrect characters to field names in a database,
we could add code to check that field names in feature classes never
change from their initial values.  We can make sure that a library
call to add a field to a feature class adds it to a database of all
field names collected at the start of testing, and collect the set of
fields in each feature class file at the beginning of each test,
storing these in a dictionary.  This example code shows two more
features of TSTL: TSTL supports {\tt init:} code in the preamble,
which is called before each test starts.  Ending a line in a backslash
indicates the action continues on the next line of the file.

{\scriptsize
\begin{code}
init: <fieldnames> = getAllFieldNames(getFeatureClasses())

\{ExecuteError\} not (<fc,1> in <hascursor>) -> \\
   AddField\_management(<fc>,<fieldname>,<fieldtype>); \\
   <fieldnames> [<fc,1>].add([<fieldname,1>])

\{IOError\} <fieldlist> := ListFields(<fc>) \\
  => sorted(<fieldlist,1>) == sorted(list(<fieldnames> [<fc,1>]))
\end{code}
}

Note the additional guard on adding fields --- we have discovered that
adding a field to a feature class that has any database cursors active
tends to crash ArcPy.  For more complicated post-conditions, the construct {\tt
  pre<(expr)>} allows access to values of expressions from before the
action was executed, as a further convenience for expressing properties.

When an assertion is an invariant on all post-action states, it can be
included in the preamble.  To check field names we would write {\tt
  property: sorted(ListFields(<fc>))==sorted(list(<fieldnames>
  [<fc,1>]}.

This property checks all feature classes, not just those whose
fields are extracted.  The advantage is that the property will
catch problems even if we never construct a {\tt fieldlist}; the
disadvantage is that testing slows to check all field names for all
feature classes, after every action.

\subsection{Differential Testing Support}
\label{sec:differential}

Another useful feature of TSTL is the ability to create
\emph{reference} pools, where every action on pool values is mirrored
by an action on a reference version of that pool.  This makes it
possible to perform differential testing \cite{Differential} on a
per-pool basis, rather than at the whole-system level, allowing
complex partial specifications.  The idea is that the behavior of some
pool (which could be the entire SUT state, in the extreme) can be
compared to a reference implementation that provides the same
observable behavior.  The classic example is the idea of testing a C
compiler by checking that the output of running a deterministic, well-defined
C program compiled under two different compilers (or optimization
levels) is the same.  If the systems differ, one must be incorrect.  A
simpler example is checking a set-like data structure against a
well-tested reference implementation.  The TSTL distribution includes
an example where an AVL tree implementation is checked against the
Python set implementation.

Differential testing can also be useful for applications other than
testing complete systems against each other; the SUT may provide
different implementations of essentially the same functionality,
serving as a reference for itself (compilers are often tested against
theiw own code, with optimization turned off). In ArcPy we may want to
ensure that operations are deterministic: no GIS operations produce
different results, given the same underlying starting feature class
data.  Assuming in raw Python in the preamble we have defined {\tt
  identityFunction} as an identity function and {\tt copyFCName} as a
function that takes a feature class name and transforms it into a
generated name for a reference copy of the feature class, the
following mirrors all actions on feature classes on a reference copy,
and checks that the feature class and its reference always have the
same fields.

{\scriptsize
\begin{code}
pools:
  <basefc> 2 CONST
  <fc> 2 CONST REF

<basefc> := <[``d1.shp'', ``d2.shp'', ``d3.shp'']>; \\
  CopyFeatures\_management(<basefc,1>,copyFCName(<basefc,1>)
<fc> := identityFunction(<basefc>)
\{IOError\} <fieldlist> := ListFields(<fc>)

references:
  identityFunction ==> copyFCName

compares:
  ListFields
\end{code}
}

When instantiating the action templates, TSTL always produces a copy
of every action containing any reference pool values.  First, the pools
are replaced with their reference copies; second, all the
syntactic transformations (which can include arbitrary Python regular
expressions) in the {\tt references} declaration are applied, to
produce the appropriate Python expression to evaluate to perform the
action using the reference implementation.
Finally, if any string matches a regular expression in a {\tt
  compares} declaration, the return values or assigned values in the
action are compared with those for the reference version, and a fault
is raised if the results are not equivalent.  In the
ArcPy case, if our {\tt copyFCName} is correctly defined, we can even
check that behavior is equivalent for different underlying data file
formats for feature classes, by treating, e.g., shapefiles as a
reference for a personal geodatabase.  In addition to such simple (and
modular) reference checking, TSTL allows properties to use the value
of a reference pool, with syntax like:

{\scriptsize
\begin{code}
property:  str(<expr>) == str(REF:<expr,1>)
\end{code}
}

\subsection{How to Build a TSTL Harness}

In the introduction, we noted that one problem with automatic
extraction of harnesses by testing tools is that in order to effectively test
complex systems, it is important to incrementally build testing
capability.  Often, as with software development, understanding the
effort as it slowly increases in scope is essential.  TSTL
naturally supports this methodology.  The ArcPy harness, though
complex, was developed by starting with a small number of ArcPy
functions, and determining their parameters.  These functions were
chosen because they were involved in unusual or problematic behavior
experienced by the authors of the paper.  Once
functions have been chosen, and their parameters are known,
developing a harness can often be a clean, iterative process:

\begin{enumerate}
\item Choose a new function (or set of related functions) to include
  in the harness.
\item Determine all parameter types for these functions.
\item If there is no pool that can produce these types, determine how
  to produce these types, and add pools and pool initialization actions for
  those pools.  This may require adding some additional
  functions (in which case, go to step 1 and start with those
  functions, recursively).
\item Add an action to call the function(s) being added.  If relevant,
  allow any expected exceptions, guards, and post-conditions to check.
\item Run testing, examine code coverage and failures to evaluate the
  added harness features, and repeat from step 1.
\end{enumerate}

These steps, combined with occasional refactoring or generalization of
parts of the harness, can effectively test even a large library, while
maintaining tester understanding and control.   In the ArcPy test
harness development, most of the effort was spent in this cycle, with
major exceptions being the implementation of a method allowing users
to provide their own GIS data as a basis for testing, and efforts to
improve the TSTL tool infrastructure to support testing a complex
application in a Windows environment.

%\subsection{Defining Complex Properties}

Note that because TSTL defines the structure of a potentially infinite
number of tests, users are expected to define correctness of a system
via \emph{property-based testing} \cite{ClaessenH00}.  As in
model checking, the correctness of the system is not specified via users
determining the specific output for each test sequence, but by
defining general properties over all executions.  TSTL includes
idiomatic support for common forms of property-based specification,
including invariants, post-conditions on actions, and comparison with
the outputs of a reference implementation \cite{Differential}.

\subsection{TSTL and Other Languages}

At heart, the TSTL ``language'' consists of the syntax and semantics for pools,
nondeterministic choice\footnote{Nondeterministic choice \cite{EWD:Discipline,McCarthy,Floyd,woda12} is both
  inherent in the notion of a TSTL action, and represented more
  concretely by the
  syntactic sugar of the {\tt <[...]>} notations.  Arguably, building
  the language and semantics around nondeterministic choice to
  represent a transition system/state space is the core idea of TSTL.},
guards, pre-conditions, pre- values, and a few other elements on top of
an existing language:  the abstract basis of TSTL has a conceptually
small footprint.

While the implementation effort to compile to an SUT interface in a
different language is considerable, there is no difficulty (beyond
engineering effort) in mapping TSTL to languages other than Python.
In one summer, an advanced high school student was able to produce a
working version for Java \cite{TSTLJava}.  Implementing TSTL for
Scala, Ruby, or even C/C++ would require effort but no research
breakthroughs.  As with the Python and Java TSTL implementations,
there is no need (due to the intentional template structure) to even
parse the underlying language.  The primary development effort is
translating the TSTL ``runtime'' of utility functions to a new
language.  For Scala, Ruby, or Swift we believe this would be quite
trivial.  For C/C++ the relative lack of functional language features
would be frustrating, but certainly not a blocking difficult.
The ideas behind TSTL are abstract and generally
applicable, even if the current implementation is built for Python.


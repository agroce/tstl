\section{The TSTL Harness Language}

The TSTL compiler takes as input a harness template file, and produces
as output a Python class file that implements an interface other tools
can use to perform testing on the SUT via an SUT-independent interface.

The harness in Figure \ref{fig:MakeFeatureLayer} shows many of the
basic features of TSTL.  The basic structure of a TSTL harness
consists of three parts, usually written in order.  First, harness
code prefixed by an {\tt @} or enclosed in {\tt <@ @>} is treated as
raw Python code, and essentially not interpreted by the TSTL
compiler.  This code is reproduced almost literally in the output
file\footnote{TSTL does have to scan {\tt import}s to re-load modules, and also pre-processes function
  definitions to support
pre- and post-conditions.}.  Second, there is a preamble that almost
always defines a set of \emph{value pools} for use in testing, but
also may include information on logging, correctness properties,
source code locations for code coverage analysis, and other basic
information that applies to the entire harness.  Finally, the bulk of
a TSTL harness (and the only non-optional element) is a set of
\emph{action definitions}.  Actions are the possible steps to be taken in
testing, and define the set of possible tests.

The original version of TSTL \cite{NFM15} required cumbersome use of Python
functions to implement many simple operations, including guards.  Current TSTL extends
the language to make it possible to define very complex test spaces
using only pools and actions, with helper functions only required for
the usual reasons of abstraction and readabiliy.

\subsection{The Essentials of Pools and Actions}

In TSTL, tests usually consist of assignments to value pools and uses of the
values in those pools.  In order to make the core ideas clear,
consider part of Figure \ref{fig:MakeFeatureLayer}, defining how to
generate values used in SQL where clauses.  The following, by itself,
is a valid TSTL harness (albeit one that cannot discover any
interesting faults):

{\scriptsize
\begin{code}
pools:
  <val> 2 CONST
\vspace{0.05in}
actions:
\vspace{0.05in}
<val> := <1..10>
<val> = <val> + 1
\end{code}
}

There is only one pool, named {\tt val} (labeled as {\tt CONST} to
indicate that its value does not change unless it appears on the left
hand side of an assignment).  The pool has room to store
two values.  The state of the SUT is defined by the state of all
pools.  Initially, all pools are set to a special value ({\tt None}) indicating the
pool has not been initialized.  Pools can be thought of as 
normal Python variables, for the most part, so we can think of the
{\tt val} pools as variables {\tt val0} and {\tt val1}.

Actions that include the {\tt :=} form of assignment (a TSTL, not
Python, operation) initialize pool values.  When {\tt <val>} appears
in an action, that represents all possible pool locations with that
name:  for our simple example, either {\tt val0} or {\tt val1}.  An
integer range is represented by {\tt <i..j>}, and TSTL expands such
ranges to produce an action with each possible choice. The
first line in the actions section of this harness translates to 20
different possible actions:

\begin{figure}
\includegraphics[width=\columnwidth]{states}
\caption{Constraints on actions in a test, based on pool states}
\label{fig:poolacts}
\end{figure}

{\scriptsize
\begin{code}
val0 = 1
val0 = 2 ...
val0 = 10
val1 = 1 ...
val1 = 10
\end{code}
}

From the initial state of the system, only these 20 actions are
\emph{enabled}.  The first concept that is essential to understanding
TSTL semantics is that at any state of the
system, the only actions that are enabled are those that do not
\emph{use} any non-initialized pool values.  Any appearance of a pool
value is a use, with the single exception of the left-hand-side of a
{\tt :=} initialization.  The second concept is that a value
cannot be initialized (appear on the lhs of {\tt :]}) until after at
least one action
that uses it has been executed.  Figure \ref{fig:poolacts} shows the
consequences of these rules for the simple value assignment harness
above.  The nodes in the graph are labeled with {\tt state(val0),
  state(val1)}, where state is
either {\tt None} (uninitialized), {\tt Unused} (initialized
but never used) or {\tt Used} (initialized and used at least once).
Starting from the initial state {\tt (None, None)}, a valid test is any path
through the graph. 

Tests that can be produced by this harness include, therefore,
sequences like  {\tt val0 = 3; val0 = val0 + 1; val1 = 4; val1 = val0
  + 2} and {\tt val1 = 10; val0 = 6; val0 = val1 + 1; val0 = 2; val1 =
  15}.  However, {\tt val0 = val0 + 1; val0 = 2} and
{\tt val0 = 1; val1 = 1; val1 = 4} are not valid tests, because they
either use an uninitialized pool value, or re-initialize an unused
pool (a clearly useless action sequence).

\subsection{Other Core Language Features}

The example TSTL harness in Figure \ref{fig:MakeFeatureLayer} shows a
few other important core elements of TSTL.  First, choice templates
are not limited to integer ranges, but can include arbitrary items in
a list, e.g, {\tt <fc> := <["d1.shp", "d2.shp", "d3.shp"]>}.
Normally, when an action raises an uncaught exception, this is
considered a test failure.  Prefixing an action with a set of
exception names in curly braces (e.g., {\tt \{IOError\}}) indicates
that some exceptions are expected, and do not indicate a failure.
More critically, actions can be prefixed by arbitrary guards, using
the syntax {\tt guard -> action}.  The simple ArcPy harness chooses
field names for SQL by first extracting a list of all fields in some
feature class.  It then allows a field name to be chosen by taking the
name of the first field in the list.  However, since the harness also
allows the list of fields to be stepped through by discarding the
initial element, the name extraction has to be guarded to ensure that
tests won't try to extract names from an empty field list:  {\tt
  len(<fieldlist,1>) >= 1 -> <fieldname> := <fieldlist> [0].name}.
The {\tt <fieldlist,1>} construct, which can also be used outside of a
guard, indicates that this pool value should not be produced using
normal template expansion (instantiated as both {\tt fieldname0} and
{\tt fieldname1}) but rather than it should copy the comma indexed
appearance of that pool in each expansion (indexing starts from 1).  This makes sure the guard
is over the same pool value that is used in the action.

TSTL also supports post-conditions on actions, in the form {\tt action
  => post-condition}, where an assertion is
checked after certain actions are performed.  For example, because
some known ArcPy bugs involve addition of incorrect characters to
field names in a database, we could add code to check that all field
names in feature classes are correct.  TSTL supports {\tt init:} code
in the preamble, which is called before each test starts.  We can then
make sure that a library call to add a field to a feature class adds
it to a database of all field names collected at the start of testing,
and check the set of values returned by 
at the beginning of each test, and store these in a dictionary.  We
can update the data if, for example, we add a field, and then check
that all field names are correct every time we access a feature
classes' fields.  Ending a line in {\tt \\} indicates the action
continues on the next line of the file.

{\scriptsize
\begin{code}
init: <fieldnames> = getAllFieldNames(getFeatureClasses())

\{ExecuteError\} not (<fc,1> in <hascursor>) -> \\
   AddField\_management(<fc>,<fieldname>,<fieldtype>); \\
   <fieldnames> [<fc,1>].add([<fieldname,1>])

\{IOError\} <fieldlist> := ListFields(<fc>) \\
  => sorted(<fieldlist,1>) == sorted(list(<fieldnames> [<fc,1>]))
\end{code}
}

Note the additional guard on adding fields --- we have discovered that
adding a field to a feature class that has any database cursors active
tends to crash ArcPy.  In post-conditions, the construct {\tt
  pre<(expr)>} allows access to values of expressions from before the
action is executed, as a further convenience.

When a correctness property is more general than a post-condition on a
single action, it can also be included in the preamble, to be checked
after every action.  Our field name property could therefore also be
stated as {\tt property: sorted(ListFields(<fc>)) ==
  sorted(list(<fieldnames> [<fc,1>]}.  The advantage is that this will
catch problems even if we never construct a {\tt fieldlist}; the
disadvantage is that testing slows to check all field names for all
feature classes, after every action.
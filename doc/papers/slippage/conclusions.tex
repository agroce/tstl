\section{Conclusions and Future Work}

In this paper we describe our first efforts to further investigate the
phenomenon of slippage, where reducing the size of a failing test also changes
the reason that the test fails.  While generally noted as a problem,
we show that slippage can also be considered an opportunity to extract
more distinct faults from a single failing test.  We present two
approaches, {\tt comb-block} and {\tt multi-ddmin}, that modify
traditional delta debugging to return multiple reduced tests.
Preliminary experiments on a Python AVL tree and on Mozilla's SpiderMonkey
JavaScript engine show that these algorithms can, at relatively
modest cost, significantly improve the number of distinct faults
detected based on a single failing test.

As future work, we plan to determine the degree to which harmful
slippage is a real problem (not solved by simple heuristics) in
real-world testing efforts, and further investigate the causes of such
slippage.  For example, is slippage overwhelmingly a matter of
hard-to-detect faults reducing to easy-to-detect faults?  We also plan to further refine and evaluate
our slippage mitigation approaches.  Most importantly, we want to
investigate whether encouraging slippage (via multiple delta debugging
runs) is an efficient way to increase fault detection for a testing
effort.

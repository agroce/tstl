\section{Related Work}

The tools described here are obviously inspired by  delta-debugging
\cite{DD} and the idea that tests should not contain extraneous parts not needed to
cause test failure (or other behavior of interest \cite{icst2014,stvrcausereduce}).  Delta-debugging and slicing
\cite{TCminim} produce subsets of the
original test, but do not modify parts of the test to obtain further
simplicity.  Our work on normalization \cite{OneTest} extends this
idea to rewrite tests into a more canonical' form.  

Zhang \cite{SaiSimple} proposed an approach to semantic
test simplification that is also able to modify, rather
than simply remove, portions of a test.  However, Zhang's simplification
operates directly over a fragment of Java, rather than
using an abstraction of test actions, with limited power: no new methods can be
invoked, statements cannot be re-ordered, and no new values are used.
It also does not even force a test to use
fixed variable names when variable name is irrelevant.  CReduce
\cite{CReduce} performs some simple normalization as part of its
test reduction for C code.  By writing a TSTL harness that is in the form of
constructor calls to create an AST, TSTL can reduce and normalize hierarchically
structured input data in ways similar to CReduce and Hierarchical
Delta Debugging \cite{HDD}.  The methods for avoiding slippage are
based on both our recent work \cite{slippage} and older heuristics for
avoiding test slippage \cite{ICSEDiff}.

The most closely related work to our test generalization \cite{OneTest} is Pike's
SmartCheck \cite{SmartCheck}.  SmartCheck works with algebraic data in
Haskell, and is an alternative approach to reduction
and generalization.  The only other work we are aware of that is
similar to generalization concerns causality in
model checking counterexamples \cite{FreeWill,MakeMost,SPIN03}.
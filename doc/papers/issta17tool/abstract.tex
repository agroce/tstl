Automated test generation tools (we hope) produce failing tests from time to time.  In a world of fault-free code this would not be true, but in such a world we would not need automated test generation tools.  Failing tests are generally speaking the most valuable products of the testing process, and users need tools that extract their full value.  This paper describes the tools provided by the TSTL testing language for making use of tests (which are not limited to failing tests).  In addition to the usual tools for simple delta-debugging and executing tests as regressions, TSTL provides tools for 1) minimizing tests by criteria other than failure, such as code coverage, 2) normalizing tests to achieve further reduction and canonicalization than provided by delta-debugging, 3) generalizing tests to describe the neighborhood of similar tests that fail in the same fashion, and 4) avoiding slippage, where delta-debugging causes a failing test to change underlying fault.  These tools can be accessed both by easy-to-use command-line tools and via a powerful API that supports more complex custom test manipulations.
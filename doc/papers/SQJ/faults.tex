\section{Faults Discovered}

Thus far, our fault-finding process has focused on crashes.  Because
test cases that cause crashes stop the testing process, it is critical
to identify and avoid behavior that causes crashes before proceeding
to add further correctness properties.  Additionally, other than
hard-to-detect data corruption, crashes may be the most frustrating
faults for a developer to fix.  When ArcGIS or a standalone ArcPy
script causes a system crash, there is no readable error message, or
symptom in incorrect data to use in debugging the program.
Understanding system core dumps or analyzing the XML logs produced by
the ArcGIS engine is difficult for end users.  It would be ideal to
fix all ArcPy crashes by (at least) changing the library behavior to
issue an error message on invalid calls, but in the absence of bug
fixes, it is helpful to identify the root causes of various crashes
for users.  These test cases all involve calls that, as far as we
can tell, is not explicitly 

Figures \ref{fault1}-\ref{fault4} show four test cases that result in
an ArcPy crash (the Python interpreter stops functioning, terminating
the test prematurely).  If executed in ArcGIS, these fragments will
crash ArcGIS.  Because these test cases are 1-minimal \cite{DD},
normalized, and generalized, we are able to describe in some detail
the general sequence of actions resulting each crash.

\subsection{First Crash Fault}

ArcPy crashes when the feature class from which a layer is produced is
deleted, and the layer is used in a {\tt SelectLayer} call (this
version shows an attribute-based selection, but location selection
will cause the same problem): (Figure \ref{fault1}).  The underlying issue seems to be that
while operations on a deleted feature class properly notify a user the
feature class does not exist, ArcPy or ArcGIS does not track that
layers depending on that feature should also be deleted/invalidated
when the feature class is deleted.

\begin{figure}
{\scriptsize 
\begin{code}
shapefilelist0 = glob.glob("C:\\Arctmp\\*.shp")                                           \textcolor{black!60}{\# STEP 0}
\textcolor{black!60}{\#[}
shapefile0 = shapefilelist0 [0]                                           \textcolor{black!60}{\# STEP 1}
newlayer0 = "l1"                                                          \textcolor{black!60}{\# STEP 2}
\textcolor{black!60}{\#  or newlayer0 = "l2" }
\textcolor{black!60}{\#  or newlayer0 = "l3" }
\textcolor{black!60}{\#  swaps with steps 3 4 5 6 7}
\textcolor{black!60}{\#] (steps in [] can be in any order)}
\textcolor{black!60}{\#[}
featureclass0 = shapefile0                                                \textcolor{black!60}{\# STEP 3}
\textcolor{black!60}{\#  swaps with step 2}
fieldname0 = "newf1"                                                      \textcolor{black!60}{\# STEP 4}
\textcolor{black!60}{\#  or fieldname0 = "newf2" }
\textcolor{black!60}{\#  or fieldname0 = "newf3" }
\textcolor{black!60}{\#  swaps with steps 2 8}
selectiontype0 = "SWITCH\_SELECTION"                                       \textcolor{black!60}{\# STEP 5}
\textcolor{black!60}{\#  or selectiontype0 = "NEW\_SELECTION" }
\textcolor{black!60}{\#  or selectiontype0 = "ADD\_TO\_SELECTION" }
\textcolor{black!60}{\#  or selectiontype0 = "REMOVE\_FROM\_SELECTION"}
\textcolor{black!60}{\#  or selectiontype0 = "SUBSET\_SELECTION"}
\textcolor{black!60}{\#  or selectiontype0 = "CLEAR\_SELECTION"   }
\textcolor{black!60}{\#  swaps with steps 2 8}
op0 = ">"                                                                 \textcolor{black!60}{\# STEP 6}
\textcolor{black!60}{\#  or op0 = "<" }
\textcolor{black!60}{\#  swaps with steps 2 8}
val0 = "100"                                                              \textcolor{black!60}{\# STEP 7}
\textcolor{black!60}{\#  or val0 = "1000" }
\textcolor{black!60}{\#  swaps with steps 2 8}
\textcolor{black!60}{\#] (steps in [] can be in any order)}
arcpy.MakeFeatureLayer\_management(featureclass0, newlayer0)                                             \textcolor{black!60}{\# STEP 8}
\textcolor{black!60}{\#  swaps with steps 4 5 6 7}
arcpy.SelectLayerByAttribute\_management(newlayer0,selectiontype0,
   ' "'+fieldname0+'" '+op0+val0)                                         \textcolor{black!60}{\# STEP 9}
arcpy.Delete\_management(featureclass0)                                    \textcolor{black!60}{\# STEP 10}
arcpy.SelectLayerByAttribute\_management(newlayer0,selectiontype0,
   ' "'+ fieldname0+'" '+op0+val0)                                        \textcolor{black!60}{\# STEP 11}
\end{code}
}
\caption{Deleting a feature class does not invalidate or delete layers that depend on it.}
\label{fault1}
\end{figure}

\subsection{Second Crash Fault}

ArcPy crashes when asked to compute statistics (wth the {\tt FIRST} or
{\tt LAST} statistics types) over a field of a layer, when that field
has been deleted from the underlying feature class:  Figure
\ref{fault2}.  This is possibly related to the first crash fault:
ArcGIS does not seem to properly propagate changes to an underlying
feature class to layers created on that feature class using a {\tt
  MakeFeatureLayer} call.

\begin{figure}
{\scriptsize 
\begin{code}
shapefilelist0 = sorted(glob.glob(arcpy.env.workspace + "\\*.shp"))                   \# STEP 0
\#[
shapefile0 = shapefilelist0 [0]                                                      \# STEP 1
newlayer0 = "l1"                                                                     \# STEP 2
\#  or newlayer0 = "l2" 
\#  or newlayer0 = "l3" 
\#  swaps with step 3
\#] (steps in [] can be in any order)
\#[
featureclass0 = shapefile0                                                           \# STEP 3
\#  swaps with step 2
classorlayer0 = newlayer0                                                            \# STEP 4
\#  swaps with steps 10 11 12
fieldtype0 = "DATE"                                                                  \# STEP 5
\#  or fieldtype0 = "TEXT" 
\#  or fieldtype0 = "FLOAT" 
\#  or fieldtype0 = "DOUBLE" 
\#  or fieldtype0 = "SHORT" 
\#  or fieldtype0 = "LONG" 
\#  swaps with steps 10 11 12
fieldname0 = "newf1"                                                                 \# STEP 6
\#  swaps with steps 10 11 12
op0 = ">"                                                                            \# STEP 7
\#  or op0 = "<" 
\#  or op0 = "<=" 
\#  or op0 = ">=" 
\#  or op0 = "=" 
\#  swaps with steps 10 11 12 14
val0 = "100"                                                                         \# STEP 8
\#  or val0 = "1000" 
\#  swaps with steps 10 11 12 14
stattable0 = arcpy.env.workspace + "\\stats.dbf"                                      \# STEP 9
\#  swaps with steps 10 11 12 14
\#] (steps in [] can be in any order)
\#[
fieldlist0 = arcpy.ListFields(featureclass0)                                         \# STEP 10
\#  swaps with steps 4 5 6 7 8 9 14
stattype0 = "FIRST"                                                                  \# STEP 11
\#  or stattype0 = "LAST" 
\#  swaps with steps 4 5 6 7 8 9
statfields0 = []                                                                     \# STEP 12
\#  swaps with steps 4 5 6 7 8 9
\#] (steps in [] can be in any order)
\#[
statfields0.append([fieldname0,stattype0])                                           \# STEP 13
arcpy.AddField\_management(featureclass0,fieldname0,fieldtype0); report()             \# STEP 14
\#  swaps with steps 7 8 9 10
\#] (steps in [] can be in any order)
fieldname0 = fieldlist0 [0].name \# STEP 15
arcpy.MakeFeatureLayer\_management(featureclass0,newlayer0,
   where\_clause=' "' + fieldname0 + '" ' + op0 + val0); report()                     \# STEP 16
fieldname0 = "newf1"                                                                 \# STEP 17
arcpy.DeleteField\_management(featureclass0,fieldname0); report()                     \# STEP 18
arcpy.Statistics\_analysis(classorlayer0,stattable0,statfields0); report()            \# STEP 19
\end{code}
}
\caption{Deleting a field then computing statistics on it causes a crash.}
\label{fault2}
\end{figure}

\begin{figure}
{\scriptsize 
\begin{code}
shapefilelist0 = sorted(glob.glob(arcpy.env.workspace + "\\*.shp"))                   \# STEP 0
shapefile0 = shapefilelist0 [0]                                                      \# STEP 1
featureclass0 = shapefile0                                                           \# STEP 2
\#[
classorlayer0 = featureclass0                                                        \# STEP 3
fieldtype0 = "DOUBLE"                                                                \# STEP 4
\#  or fieldtype0 = "TEXT"
\#  or fieldtype0 = "FLOAT"
\#  or fieldtype0 = "SHORT"
\#  or fieldtype0 = "LONG"
\#  or fieldtype0 = "DATE"
\#  swaps with step 6
fieldname0 = "newf1"                                                                 \# STEP 5
\#  or fieldname0 = "newf3"
\#  swaps with steps 6 8
\#] (steps in [] can be in any order)
\#[
insertcursor0 = arcpy.InsertCursor(classorlayer0)                                    \# STEP 6
\#  swaps with steps 4 5
arcpy.AddField\_management(featureclass0,fieldname0,fieldtype0); report()             \# STEP 7
\#] (steps in [] can be in any order)
fieldname0 = "newf2"                                                                 \# STEP 8
\#  or fieldname0 = "newf3"
\#  swaps with step 5
arcpy.AddField\_management(featureclass0,fieldname0,fieldtype0); report()             \# STEP 9
insertcursor0 = arcpy.InsertCursor(classorlayer0)                                    \# STEP 10
\end{code}
}
\caption{Creating two insert cursors on a layer, after adding two fields to the feature class underlying it, causes a crash.}
\label{fault3}
\end{figure}

\begin{figure}
{\scriptsize
\begin{code}
shapefilelist0 = sorted(glob.glob(arcpy.env.workspace + "\\*.shp"))                   \# STEP 0
\#[
shapefile0 = shapefilelist0 [0]                                                      \# STEP 1
newlayer0 = "l1"                                                                     \# STEP 2
\#  or newlayer0 = "l2" 
\#  swaps with step 3
\#] (steps in [] can be in any order)
\#[
featureclass0 = shapefile0                                                           \# STEP 3
\#  swaps with step 2
classorlayer0 = newlayer0                                                            \# STEP 4
\#  or classorlayer0 = featureclass0 
\#  or (
\#      newlayer0 = "l3"  ;
\#      classorlayer0 = newlayer0 
\#     )
fieldtype0 = "FLOAT"                                                                 \# STEP 5
\#  or fieldtype0 = "DOUBLE" 
\#  or fieldtype0 = "SHORT" 
\#  or fieldtype0 = "LONG" 
fieldname0 = "newf1"                                                                 \# STEP 6
\#  or fieldname0 = "newf3" 
\#  swaps with step 11
op0 = ">"                                                                            \# STEP 7
\#  or op0 = "<" 
\#  or op0 = "<=" 
\#  or op0 = ">=" 
\#  or op0 = "=" 
val0 = "10"                                                                          \# STEP 8
\#  or val0 = "20" 
\#  or val0 = "30" 
\#  or val0 = "100" 
\#  or val0 = "1000" 
\#] (steps in [] can be in any order)
\#[
whereclause0 = '"' + fieldname0 + '" ' + op0 + str(val0)                             \# STEP 9
arcpy.AddField\_management(featureclass0,fieldname0,fieldtype0); report()             \# STEP 10
\#] (steps in [] can be in any order)
\#[
fieldname0 = "newf2"                                                                 \# STEP 11
\#  or fieldname0 = "newf3" 
\#  swaps with step 6
arcpy.MakeFeatureLayer\_management(featureclass0,newlayer0,where\_clause=whereclause0); report()   \# STEP 12
\#] (steps in [] can be in any order)
searchcursor0 = arcpy.SearchCursor(classorlayer0,whereclause0)                       \# STEP 13
\#  or searchcursor0 = arcpy.SearchCursor(classorlayer0) 
arcpy.AddField\_management(featureclass0,fieldname0,fieldtype0); report()             \# STEP 14
\#  or (
\#      fieldtype0 = "TEXT"  ;
\#      arcpy.AddField\_management(featureclass0,fieldname0,fieldtype0); report() 
\#     )
\#  or (
\#      fieldtype0 = "DATE"  ;
\#      arcpy.AddField\_management(featureclass0,fieldname0,fieldtype0); report() 
\#     )
srow0 = searchcursor0.next()                                                         \# STEP 15
\#  or srow1 = searchcursor0.next() 
\#  or srow2 = searchcursor0.next()
\end{code}
}
\caption{Advancing a search cursor on a layer, after adding a field to
  the underlying feature class twice, causes a crash.}
\label{fault4}
\end{figure}
\subsection{Sandboxed Test Case Execution}

Previous use of TSTL had been limited to testing systems where failure
resulted in an uncaught exception or a bad return value from a call,
at worst.  With ArcPy, however, it is very common for a failure to
cause a crash, killing not only ArcPy but the Python environment
running the test case.

We added two features to TSTL's test generators to handle this
problem.  First, we modified the random tester to record each action
to a test case log \emph{before the action is performed}, in order to recover a crashing
test after the test generator terminates abnormally.  After further experimentation,
we discovered that recording just the current test case was not
sufficient; some ArcPy failures required maintaining a history of all
executed actions, since the corruption carried across reimports of the
Python module.  This required us to add a new type of special action
to the TSTL interface, the {\tt restart} action.  When a {\tt restart}
appears in a test, it ``restarts'' the test.  The assumption until we
began testing ArcPy was that all actions before such restarts could be
removed from a test case.  

The second new feature required for effective sandboxing was a
function to enable running a TSTL case in its own Python subprocess,
to allow test reduction, normalization, and generalization (see below)
even for crashing test cases.  It was not neccessary to modify the
TSTL interface's API, as the relevant functions already took an
arbitrary function as reduction predicate, allowing us to simply
produce a ``sandboxed'' version of replay and pass that to TSTL's {\tt
  reduce}, {\tt normalize} etc. calls.  The sandboxing in this case is
minimal: the only resource limitation is that the sandboxed execution
has its own Python process and writes to a temporary copy of the
ArcGIS workspace, but in principle the approach can be extended to
allow more restrictive test case jails in TSTL, including execution in
a virtual machine.

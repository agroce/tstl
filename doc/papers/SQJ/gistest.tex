%%%%%%%%%%%%%%%%%%%%%%% file template.tex %%%%%%%%%%%%%%%%%%%%%%%%%
%
% This is a general template file for the LaTeX package SVJour3
% for Springer journals.          Springer Heidelberg 2010/09/16
%
% Copy it to a new file with a new name and use it as the basis
% for your article. Delete % signs as needed.
%
% This template includes a few options for different layouts and
% content for various journals. Please consult a previous issue of
% your journal as needed.
%
%%%%%%%%%%%%%%%%%%%%%%%%%%%%%%%%%%%%%%%%%%%%%%%%%%%%%%%%%%%%%%%%%%%
%
% First comes an example EPS file -- just ignore it and
% proceed on the \documentclass line
% your LaTeX will extract the file if required
\begin{filecontents*}{example.eps}
%!PS-Adobe-3.0 EPSF-3.0
%%BoundingBox: 19 19 221 221
%%CreationDate: Mon Sep 29 1997
%%Creator: programmed by hand (JK)
%%EndComments
gsave
newpath
  20 20 moveto
  20 220 lineto
  220 220 lineto
  220 20 lineto
closepath
2 setlinewidth
gsave
  .4 setgray fill
grestore
stroke
grestore
\end{filecontents*}
%
\RequirePackage{fix-cm}
%
%\documentclass{svjour3}                     % onecolumn (standard format)
%\documentclass[smallcondensed]{svjour3}     % onecolumn (ditto)
\documentclass[smallextended]{svjour3}       % onecolumn (second format)
%\documentclass[twocolumn]{svjour3}          % twocolumn
%
\smartqed  % flush right qed marks, e.g. at end of proof
%
\usepackage{graphicx}
\usepackage{code}
\usepackage{xcolor}
%
% \usepackage{mathptmx}      % use Times fonts if available on your TeX system
%
% insert here the call for the packages your document requires
%\usepackage{latexsym}
% etc.
%
% please place your own definitions here and don't use \def but
% \newcommand{}{}
%
% Insert the name of "your journal" with
% \journalname{myjournal}
%
\begin{document}

\title{Extensible and Usable Testing for Geographic Information System Automation}
%\subtitle{Do you have a subtitle?\\ If so, write it here}

%\titlerunning{Short form of title}        % if too long for running head

\author{Josie Holmes         \and
  Alex Groce           \and
  James O'Brien
}

%\authorrunning{Short form of author list} % if too long for running head

\institute{Alex Groce \at
              School of Electrical Engineering and Computer Science\\
              Oregon State University\\
              \email{agroce@gmail.com}           %  \\
%             \emph{Present address:} of F. Author  %  if needed
           \and
           Josie Holmes \at
             Department of Geography\\
             Pennsylvania State University\\
           \email{jdh396@psu.edu}
           \and
           James O'Brien \at
          Risk Frontiers\\
           Macquarie University\\
          \email{James.OBrien@mq.edu.au}
}

\date{Received: date / Accepted: date}
% The correct dates will be entered by the editor


\maketitle

\begin{abstract}
Automated test generation tools (we hope) produce failing tests from time to time.  In a world of fault-free code this would not be true, but in such a world we would not need automated test generation tools.  Failing tests are generally speaking the most valuable products of the testing process, and users need tools that extract their full value.  This paper describes the tools provided by the TSTL testing language for making use of tests (which are not limited to failing tests).  In addition to the usual tools for simple delta-debugging and executing tests as regressions, TSTL provides tools for 1) minimizing tests by criteria other than failure, such as code coverage, 2) normalizing tests to achieve further reduction and canonicalization than provided by delta-debugging, 3) generalizing tests to describe the neighborhood of similar tests that fail in the same fashion, and 4) avoiding slippage, where delta-debugging causes a failing test to change underlying fault.  These tools can be accessed both by easy-to-use command-line tools and via a powerful API that supports more complex custom test manipulations.
\keywords{Automated testing
  \and End-user developers \and Geographic Information System \and
  Testing languages \and Debugging}
% \PACS{PACS code1 \and PACS code2 \and more}
% \subclass{MSC code1 \and MSC code2 \and more}
\end{abstract}

\section{Introduction}

\subsection{Esri ArcPy}


Esri (the Environmental Systems Research Institute) is the single
largest Geographic Information System (GIS) software vendor, with about 40\%
of global market share.  Esri's ArcGIS tools are extremely widely
used for GIS analysis, in government, scientific research, commercial
enterprises, and education.  Automation of complex GIS analysis and
data management is a frequent need, and Esri has long provided tools
for programming their GIS software tools.  The current method of
choice is a Python site-package, ArcPy \cite{ArcPy}.  ArcPy is a complex library,
with dozens of classes and hundreds of functions distributed over
a variety of of toolboxes.  Most of the code involved in ArcPy
functionality is C++ for which source is unavailable (the source for
the actual ArcGIS tools); the Python source alone is over 50,000 lines
of code.

\subsection{Automated Testing for the Rest of the World}
\section{Related Work}

The tools described here are obviously inspired by  delta-debugging
\cite{DD} and the idea that tests should not contain extraneous parts not needed to
cause test failure (or other behavior of interest \cite{icst2014,stvrcausereduce}).  Delta-debugging and slicing
\cite{TCminim} produce subsets of the
original test, but do not modify parts of the test to obtain further
simplicity.  Our work on normalization \cite{OneTest} extends this
idea to rewrite tests into a more canonical' form.  

Zhang \cite{SaiSimple} proposed an approach to semantic
test simplification that is also able to modify, rather
than simply remove, portions of a test.  However, Zhang's simplification
operates directly over a fragment of Java, rather than
using an abstraction of test actions, with limited power: no new methods can be
invoked, statements cannot be re-ordered, and no new values are used.
It also does not even force a test to use
fixed variable names when variable name is irrelevant.  CReduce
\cite{CReduce} performs some simple normalization as part of its
test reduction for C code.  By writing a TSTL harness that is in the form of
constructor calls to create an AST, TSTL can reduce and normalize hierarchically
structured input data in ways similar to CReduce and Hierarchical
Delta Debugging \cite{HDD}.  The methods for avoiding slippage are
based on both our recent work \cite{slippage} and older heuristics for
avoiding test slippage \cite{ICSEDiff}.

The most closely related work to our test generalization \cite{OneTest} is Pike's
SmartCheck \cite{SmartCheck}.  SmartCheck works with algebraic data in
Haskell, and is an alternative approach to reduction
and generalization.  The only other work we are aware of that is
similar to generalization concerns causality in
model checking counterexamples \cite{FreeWill,MakeMost,SPIN03}.
\section{A Brief Primer on TSTL}

\begin{figure}
{\scriptsize
\begin{code}
$_{01}$ @import avl
$_{02}$ @import math
\vspace{0.1in}
$_{03}$ <@
$_{04}$ def it(s):
$_{05}$     l = []
$_{06}$     for i in s:
$_{07}$        l.append(i)
$_{08}$     return sorted(l)
$_{09}$ @>
\vspace{0.1in}
$_{10}$ source: avl.py
\vspace{0.1in}
$_{11}$ pool: <int> 4 CONST
$_{12}$ pool: <avl> 2 REF
$_{13}$ pool: <list> 2
\vspace{0.1in}
$_{14}$ log: 1 <avl>.inorder()
\vspace{0.1in}
$_{15}$ property: <avl>.check\_balanced()
\vspace{0.1in}
$_{16}$ <list>:=[]
$_{17}$ ~<list>.append(<int>) 
$_{18}$ <int>:= <[1..20]>
\vspace{0.1in}
$_{19}$ <avl>:=avl.AVLTree()
$_{20}$ <avl>:=avl.AVLTree(<list>)
\vspace{0.1in}

$_{21}$ ~<avl>.insert(<int>) => (len(<avl,1>.inorder()) == pre<(len(<avl,1>.inorder()))>+1) 
  or pre<(<avl,1>.find(<int,1>))>
$_{22}$ ~<avl>.delete(<int>) => (len(<avl,1>.inorder()) == pre<(len(<avl,1>.inorder()))>-1) 
  or not pre<((<avl,1>.find(<int,1>)))>
$_{23}$ ~<avl>.find(<int>)
$_{24}$ <avl>.inorder()
$_{25}$ len(<avl,1>.inorder()) > 5 -> <avl>.display()
\vspace{0.1in}
$_{26}$ reference: avl.AVLTree ==> set
$_{27}$ reference: insert ==> add
$_{28}$ reference: delete ==> discard
$_{29}$ reference: find ==> \_\_contains\_\_
$_{30}$ reference: METHOD(inorder) ==> CALL(it)
$_{31}$ reference: METHOD(display) ==> CALL(print)
\vspace{0.1in}
$_{32}$ compare: find
$_{33}$ compare: inorder
\end{code}
}
\caption{Part of a TSTL definition of AVL tree tests.}
\label{fig:example}
\end{figure}


\begin{figure}
{\scriptsize 
\begin{code}
avl1 = avl.AVLTree()  
int3 = 10  
int1 = 11  
avl1.insert(int1) 
int1 = 1  
avl1.insert(int3) 
avl1.insert(int1) 
int3 = 9  
avl1.insert(int3) 
int2 = 11  
avl1.delete(int2) 
\end{code}
}
\caption{An example TSTL-produced test.}
\label{fig:avlrun}
\end{figure}


TSTL, the Template Scripting Testing Language, \cite{NFM15,ISSTA15,tstl} is a language for defining the
structure of test cases (usually API-call sequences, but also
grammar based tests using string construction), and a set of tools for
use in generating, manipulating, and understanding those test cases.
Figure \ref{fig:example} shows a TSTL definition of tests (known as a
harness definition, or \emph{harness} for short) for a Python class
implementing AVL trees\footnote{AVL trees, named after Georgy
  Adelson-Velsky and Evgenii Landis \cite{AVL} are balanced tree
  structures similar to red-black trees; they are also frequently used as
  examples in software testing \cite{FASE,ISSRE}.} \cite{avltree}, in the latest syntax for TSTL (modified in the
course of the work described in this paper).  Given a harness like the
one in Figure \ref{fig:example}, TSTL compiles it into a Python file
defining a class that gives an interface for testing.  The class
interface provides features such as
querying the set of available testing actions, restarting a test,
replaying a test, collecting data about which lines and branches in
the source code of the SUT have been executed by each test, and other commonly
needed testing features.  The TSTL release \cite{tstl} provides
testing tools that use this interface for test generation and
debugging.

A TSTL test harness defines a set of \emph{pools} that hold values
produced and used during testing.   Using pools \cite{AndrewsTR} is a common
approach to defining API-testing sequences.  The harness also defines a set of
actions that are possible during testing, typically API calls and
assignments to pool values.  Code marked off with the {\tt @}
symbol (lines 1-9) is raw Python code.  The ability to call
arbitrary code in Python makes TSTL extremely expressive.  Other
elements of the ``preamble'' (before action definitions) are the
indication of the source files to be tested (line 10), the information
on how to produce logs of test activity for debugging (line 14), and a
correctness property to check after every test action (line 15).

In this example, there are three pools,
{\tt int}, {\tt list}, and {\tt avl}.  There are four instances of the
{\tt int} pool, which means that a test in progress can store up to 4
{\tt int}s at one time (in variables named {\tt int0}, {\tt int1},
{\tt int2}, and {\tt int3}), and two instances of the {\tt avl} and
{\tt list}
pools.  The actions defined here include setting the value of an {\tt int}
pool to any integer in the range 1-20 inclusive (line 18), setting the value of
an {\tt avl} pool to a newly constructed AVL tree (line 19), and calling an AVL
tree's {\tt insert}, {\tt delete}, {\tt find} and {\tt inorder}
methods (lines 21-24).  One line of TSTL typically defines more than one action. For
example, the line of TSTL code {\tt <avl>.find(<int>)} defines 8 actions, one
for each choice of {\tt avl} and {\tt int} pool instance:  the action set
includes {\tt avl0.find(int0)}, {\tt avl1.find(int0)}, {\tt
  avl0.find(int1)}, and so forth.  


Figure \ref{fig:avlrun} shows a valid test case produced by
running a random test generator on the TSTL-compiled interface
produced by this definition.  TSTL automatically enforces the requirement that
tests are well-formed: for example, no pool instance (such as {\tt
  avl1}) can appear in an action until it has been assigned a value,
and no pool instance that has been assigned a value can be assigned a
different value until it has been used in an action, to avoid
degenerate sequences such as {\tt int3 = 10} followed by {\tt int3 =
  4}.  Each action in a test case is called a ``step'' --- the first
step of the first test case in Figure \ref{fig:avlrun} is storing a
new AVL tree in {\tt avl1}.  Prefacing a ``use'' of a pool with a
tilde {\tt ~} indicates that use does not allow the pool to be
re-initialized; in this example, an AVLtree object cannot be replaced
with a new tree until it has been displayed or traversed.

Figure \ref{fig:example} shows some additional features
of TSTL.  The syntax {\tt guard -> action} allows {\tt action} to take
place only if {\tt guard} is an expression that evaluates to true (see
line 25).  In the example, only AVL trees of size 5 or larger are
displayed.  One feature that is not shown is the syntax for allowing
an action to throw an exception without stopping testing. This
construct, {\tt \{exception [, exception2, ...]\} action}, is used
many times in the TSTL harness, because operations in ArcPy tend to
signal failure by throwing an exception.

Because thus far our testing
has been limited to checking for crash failures and comparing outputs
between versions, we have not needed the TSTL feature for checking an
assertion after an action ({\tt action => pred}, shown on lines 21 and
22), but we expect to use it in future testing, as discussed in Section
\ref{futureoracle}.  The construct {\tt <avl,1>} is a pool use that
always expands to the same pool variable as the indexed occurrence of
the pool in the current action.  The final feature of interest here is the full
support for reference testing.  Marking a pool as {\tt REF} (line 12) creates a
second copy of each pool variable, and transforms the operations
performed on the copy according to a reference mapping, as given in
lines 26-31 of the example code.  In the example, AVL trees are
checked for behavioral equivalence to Python sets.  Lines 32 and 33
also indicate the return values from {\tt find} and {\tt inorder}
should be compared to their reference equivalents.  Section
\ref{sec:reftest} discusses the possibility of using
differential/reference testing \cite{Differential,ICSEDiff} in ArcPy
within a single version.
\section{Language Changes}

The original syntax for TSTL \cite{NFM15} used the {\tt \%} sign to
indicate TSTL constructs and pool variables.  For example, the
assignment of range values to the integer pool in the AVL example
would read:  {\tt \%INT\% := \%[1..20]\%}.  This syntax
produced code that was difficult to read and (in our opinion) unattractive.  The first
author suggested using a notation more similar to that used to
describe the grammar of TSTL itself, enclosing pools in angle
brackets.  On examination, this syntax better reflects the nature of
TSTL pools, since it resembles a BNF (Backus-Naur Form) grammar, and
pools in actions are
conceptually more similar to a grammar non-terminals than simple variables.  Because
C++ and a few other languages use angle brackets for other purposes,
and in order to avoid breaking old harnesses, TSTL continues to allow
the {\tt \%} notation, but future TSTL harnessess for Python will use
the more readable syntax.

Another language improvement inspited by development of the ArcPy harness was
extending the range construct {\tt <[start..end]>} to also express a
list of options, which can be constants or Python expressions.  An
action containing {\tt <[item1, item2, ...]>} expands to multiple
actions, each of which has the literal text of one item.  The
construct {\tt <,[item1,, item2,, ...],>} works the same way, except
that items are delimited by double-commas, allowing the items to even
be partial Python expressions containing commas, for use in,
e.g. constructing function arguments.  The second form is not used in
the ArcPy harness at this time, because it is powerful but somewhat
difficult to read.  In some cases, using multiple lines to define
actions that, in theory, could be handled with a single line is best
for readability reasons (see Section \ref{harness}).

We are also considering, as a result of the experience of developing
the ArcPy harness, moving to a more structured form for TSTL harness definitions.
The current language allows pool definitions, actions, logging code,
raw Python code, and all other TSTL elements to be freely mixed,
without any requirements as to order.  Each line must indicate if it
is not an action definition, with some prefix such as {\tt pool:},
{\tt logging:}, {\tt reference:}, etc.; in practice, however, TSTL
harnessess are always written in an ordered style, with raw code
first, then pool definitions, properties, and logging information,
followed by a long section of action definitions.  Enforcing this
would allow all pool declarations to be prefaced by a single {\tt
  pool:} line at the beginning of the pool definitions, raw Python
code to be contained in a section marked{\tt raw:}, and all other
non-action declarations to be handled in the same way. 

There is also a need for richer structure to avoid repeated
elements in action definitions.  For example, in the TSTL harness, 36
actions allow the {\tt arcpy.ExecuteError} exception to be raised,
which has to be stated for every individual action, and to avoid some
faults a large number of actions may eventually be disallowed for
feature classes or layers with active cursors\footnote{The reader
  not familiar with GIS terminology is directed to the ArcPy
  documentation \cite{ArcPy} and Esri's GIS dictionary \cite{GISDict}.}.  Introducing nested
action groups, which can share guards, allowed exceptions, and
post-conditions could make reading complex TSTL code easier.  We are
currently working to define these language changes in a way that does
not break existing TSTL code.  Discovering the need for this kind of
feature without testing a system as complex as ArcPy would be
difficult.

\section{Tool Changes}

Rather than major algorithmic innovations, the major changes to the
TSTL system required to test ArcPy were fundamentally engineering
challenges.  Only the last two changes described in this section are
concerned with novel testing algorithms and support a research
agenda.  One element that carried across all of these concerns was
fixing bugs in TSTL.  TSTL has been used in several university classes
at the graduate and undergraduate level, and used fairly extensively
in testing research, but a large number of
significant faults went undetected until we attempted to use TSTL to
test ArcPy.  One change made as a result of these problems is that we
now use TSTL to test TSTL's own API.

\subsection{Sandboxed Test Case Execution}

Previous use of TSTL had been limited to testing systems where failure
resulted in an uncaught exception or a bad return value from a call,
at worst.  With ArcPy, however, it is very common for a failure to
cause a crash, killing not only ArcPy but the Python environment
running the test case.

We added two features to TSTL's test generators to handle this
problem.  First, we modified the random tester to record each action
to a test case log \emph{before the action is performed}, in order to recover a crashing
test after the test generator terminates abnormally.  After further experimentation,
we discovered that recording just the current test case was not
sufficient; some ArcPy failures required maintaining a history of all
executed actions, since the corruption carried across reimports of the
Python module.  This required us to add a new type of special action
to the TSTL interface, the {\tt restart} action.  When a {\tt restart}
appears in a test, it ``restarts'' the test.  The assumption until we
began testing ArcPy was that all actions before such restarts could be
removed from a test case.  

The second new feature required for effective sandboxing was a
function to enable running a TSTL case in its own Python subprocess,
to allow test reduction, normalization, and generalization (see below)
even for crashing test cases.  It was not neccessary to modify the
TSTL interface's API, as the relevant functions already took an
arbitrary function as reduction predicate, allowing us to simply
produce a ``sandboxed'' version of replay and pass that to TSTL's {\tt
  reduce}, {\tt normalize} etc. calls.  The sandboxing in this case is
minimal: the only resource limitation is that the sandboxed execution
has its own Python process and writes to a temporary copy of the
ArcGIS workspace, but in principle the approach can be extended to
allow more restrictive test case jails in TSTL, including execution in
a virtual machine.


\subsection{Standalone Test Case Generation}

TSTL test cases are saved as text files, where each line of the text
file contains a string representation of an action (with a unique
such string ``name'' for each action).  This format is somewhat human
readable, but is not machine-executable without the assistance of the
TSTL interface.  These test cases are also somewhat misleading for
readers, since guards, post-conditions, reference pool actions, and
allowed exceptions do not appear in this format.

Publishing test cases (or submitting them to Esri) is impractical,
however, if it requires use of the entire TSTL toolchain.  Moreover,
making tests only work within TSTL means it is not possible to
experiment with changing test cases in ways that are not in the TSTL
action set.  It is not  even possible to add print statements to help understand
behavior, without using TSTL's complex logging mechanisms.  

In order to address these problems, we implemented a new TSTL utility
that takes a test case stored in TSTL's internal format and produces a
standalone Python file that does not require any TSTL support.  The
standalone test case generator has options to control whether the
generated test case includes actions on reference pools, property
checks, and handling for allowed exceptions (omitted only if the exception does not
actually take place when the test is replayed).  The reason for disabling
the last functionality is interesting:  most test cases that are
stored are minimized \cite{DD} test cases, from which all actions not
necessary to produce a failure or obtain desired code coverage have
been removed.  In most cases, this means that most ArcPy calls are
successful.  Adding code to handle potential exceptions makes
standalone test cases much longer and harder to read.

In the process of producing the standalone test case utility, we
introduced a more readable format for TSTL action names that is closer
to the code in standalone test cases.  This has, to our surprise, made
reading TSTL test cases in tool output, not just in standalone test
cases, much easier.  The new format has been integrated into new TSTL features.

\subsection{Regression Generation}

One difficulty for ArcPy users is ensuring that their existing scripts
and tools work on new versions of ArcGIS.  Each recent major release
(10.2 and 10.3) after ArcPy's introduction has produced some changes
in the behavior of API calls.  Detecting when such changes cause a
script to break is difficult.  A first step would be an automatic way
to find when the return values for calls differ between ArcPy
versions.
Because installing multiple versions of ArcGIS on the same system is
difficult or impossible, our method for finding differences relies on
choosing a reference version (10.3 in our current efforts), and
generating a set of standalone tests that 1) cover a large amount of
ArcPy functionality, including invalid inputs to functions and 2)
record the return values and exceptions raised by calls.  These tests
can be run on any ArcPy version, and will report differences between
the tests and version 10.3.

We generate the tests using an approach called \emph{quick testing}
\cite{icst2014,stvrcausereduce}, which takes a set of tests produced
by random testing, and applies a test case reduction algorithm
\cite{DD} to produce short tests that have the same code coverage as
the very large, highly redundant, original set of test cases.
Automatic quick-testing was added to TSTL's random test generator to
support ArcPy testing.  Combined with standalone test generation, this
allowed us to produce test cases that can be run on any version of
ArcPy, and explore a large variety of behavior of the code.  Coverage
alone, however, unlike previous quick testing efforts, is insufficient
to ensure a useful regression test.  Because coverage only considers
the Python behavior of ArcPy (since we do not have access to the
source for the ArcGIS engine), it may group behaviors that are not
similar together.  We added the ability to combine coverage
preservation with preservation of all ArcPy messages indicating a
successful GIS engine operation, after abstracting away such details
as the runtime of the operation, and so forth.

However, just producing these coverage-and-engine-behavior preserving
standalone tests is not sufficient for good version comparison, since
standalone test cases as produced only check for properties defined in TSTL.  An
additional option was added to the standalone test generator, allowing
it to record the actual return values of all calls, the set of
exceptions thrown, and so forth to more precisely record a test's
behavior on an ArcPy version.


\subsection{Test Case Normalization and Generalization}

\section{Testing from Saved States}

\section{Faults Discovered}

\begin{figure}
{\scriptsize 
\begin{code}
shapefilelist0 = glob.glob("C:\\Arctmp\\*.shp")                             \textcolor{black!60}{\# STEP 0}
\textcolor{black!60}{\#[}
shapefile0 = shapefilelist0 [0]                                           \textcolor{black!60}{\# STEP 1}
newlayer0 = "l1"                                                          \textcolor{black!60}{\# STEP 2}
\textcolor{black!60}{\#  or newlayer0 = "l2" }
\textcolor{black!60}{\#  or newlayer0 = "l3" }
\textcolor{black!60}{\#  swaps with steps 3 4 5 6 7}
\textcolor{black!60}{\#] (steps in [] can be in any order)}
\textcolor{black!60}{\#[}
featureclass0 = shapefile0                                                \textcolor{black!60}{\# STEP 3}
\textcolor{black!60}{\#  swaps with step 2}
fieldname0 = "newf1"                                                      \textcolor{black!60}{\# STEP 4}
\textcolor{black!60}{\#  or fieldname0 = "newf2" }
\textcolor{black!60}{\#  or fieldname0 = "newf3" }
\textcolor{black!60}{\#  swaps with steps 2 8}
selectiontype0 = "SWITCH\_SELECTION"                                       \textcolor{black!60}{\# STEP 5}
\textcolor{black!60}{\#  or selectiontype0 = "NEW\_SELECTION" }
\textcolor{black!60}{\#  or selectiontype0 = "ADD\_TO\_SELECTION" }
\textcolor{black!60}{\#  or selectiontype0 = "REMOVE\_FROM\_SELECTION"}
\textcolor{black!60}{\#  or selectiontype0 = "SUBSET\_SELECTION"}
\textcolor{black!60}{\#  or selectiontype0 = "CLEAR\_SELECTION"   }
\textcolor{black!60}{\#  swaps with steps 2 8}
op0 = ">"                                                                 \textcolor{black!60}{\# STEP 6}
\textcolor{black!60}{\#  or op0 = "<" }
\textcolor{black!60}{\#  swaps with steps 2 8}
val0 = "100"                                                              \textcolor{black!60}{\# STEP 7}
\textcolor{black!60}{\#  or val0 = "1000" }
\textcolor{black!60}{\#  swaps with steps 2 8}
\textcolor{black!60}{\#] (steps in [] can be in any order)}
arcpy.MakeFeatureLayer\_management(featureclass0, newlayer0)               \textcolor{black!60}{\# STEP 8}
\textcolor{black!60}{\#  swaps with steps 4 5 6 7}
arcpy.SelectLayerByAttribute\_management(newlayer0,selectiontype0,
   ' "'+fieldname0+'" '+op0+val0)                                         \textcolor{black!60}{\# STEP 9}
arcpy.Delete\_management(featureclass0)                                    \textcolor{black!60}{\# STEP 10}
arcpy.SelectLayerByAttribute\_management(newlayer0,selectiontype0,
   ' "'+ fieldname0+'" '+op0+val0)                                        \textcolor{black!60}{\# STEP 11}
\end{code}
}
\caption{Deleting a feature class does not invalidate or delete layers that depend on it.}
\label{fault1}
\end{figure}

\begin{figure}
{\scriptsize 
\begin{code}
shapefilelist0 = sorted(glob.glob(arcpy.env.workspace + "\\*.shp"))            \textcolor{black!60}{\# STEP 0}
\textcolor{black!60}{\#[}
shapefile0 = shapefilelist0 [0]                                               \textcolor{black!60}{\# STEP 1}
newlayer0 = "l1"                                                              \textcolor{black!60}{\# STEP 2}
\textcolor{black!60}{\#  or newlayer0 = "l2" }
\textcolor{black!60}{\#  or newlayer0 = "l3" }
\textcolor{black!60}{\#  swaps with step 3}
\textcolor{black!60}{\#] (steps in [] can be in any order)}
\textcolor{black!60}{\#[}
featureclass0 = shapefile0                                                    \textcolor{black!60}{\# STEP 3}
\textcolor{black!60}{\#  swaps with step 2}
classorlayer0 = newlayer0                                                     \textcolor{black!60}{\# STEP 4}
\textcolor{black!60}{\#  swaps with steps 10 11 12}
fieldtype0 = "DATE"                                                           \textcolor{black!60}{\# STEP 5}
\textcolor{black!60}{\#  or fieldtype0 = "TEXT" }
\textcolor{black!60}{\#  or fieldtype0 = "FLOAT" }
\textcolor{black!60}{\#  or fieldtype0 = "DOUBLE" }
\textcolor{black!60}{\#  or fieldtype0 = "SHORT" }
\textcolor{black!60}{\#  or fieldtype0 = "LONG" }
\textcolor{black!60}{\#  swaps with steps 10 11 12}
fieldname0 = "newf1"                                                          \textcolor{black!60}{\# STEP 6}
\textcolor{black!60}{\#  swaps with steps 10 11 12}
op0 = ">"                                                                     \textcolor{black!60}{\# STEP 7}
\textcolor{black!60}{\#  or op0 = "<" }
\textcolor{black!60}{\#  or op0 = "<=" }
\textcolor{black!60}{\#  or op0 = ">=" }
\textcolor{black!60}{\#  or op0 = "=" }
\textcolor{black!60}{\#  swaps with steps 10 11 12 14}
val0 = "100"                                                                  \textcolor{black!60}{\# STEP 8}
\textcolor{black!60}{\#  or val0 = "1000" }
\textcolor{black!60}{\#  swaps with steps 10 11 12 14}
stattable0 = arcpy.env.workspace + "\\stats.dbf"                               \textcolor{black!60}{\# STEP 9}
\textcolor{black!60}{\#  swaps with steps 10 11 12 14}
\textcolor{black!60}{\#] (steps in [] can be in any order)}
\textcolor{black!60}{\#[}
fieldlist0 = arcpy.ListFields(featureclass0)                                  \textcolor{black!60}{\# STEP 10}
\textcolor{black!60}{\#  swaps with steps 4 5 6 7 8 9 14}
stattype0 = "FIRST"                                                           \textcolor{black!60}{\# STEP 11}
\textcolor{black!60}{\#  or stattype0 = "LAST" }
\textcolor{black!60}{\#  swaps with steps 4 5 6 7 8 9}
statfields0 = []                                                              \textcolor{black!60}{\# STEP 12}
\textcolor{black!60}{\#  swaps with steps 4 5 6 7 8 9}
\textcolor{black!60}{\#] (steps in [] can be in any order)}
\textcolor{black!60}{\#[}
statfields0.append([fieldname0,stattype0])                                    \textcolor{black!60}{\# STEP 13}
arcpy.AddField\_management(featureclass0,fieldname0,fieldtype0); report()      \textcolor{black!60}{\# STEP 14}
\textcolor{black!60}{\#  swaps with steps 7 8 9 10}
\textcolor{black!60}{\#] (steps in [] can be in any order)}
fieldname0 = fieldlist0 [0].name \textcolor{black!60}{\# STEP 15}
arcpy.MakeFeatureLayer\_management(featureclass0,newlayer0,
   where\_clause=' "' + fieldname0 + '" ' + op0 + val0); report()              \textcolor{black!60}{\# STEP 16}
fieldname0 = "newf1"                                                          \textcolor{black!60}{\# STEP 17}
arcpy.DeleteField\_management(featureclass0,fieldname0); report()              \textcolor{black!60}{\# STEP 18}
arcpy.Statistics\_analysis(classorlayer0,stattable0,statfields0); report()     \textcolor{black!60}{\# STEP 19}
\end{code}
}
\caption{Deleting a field then computing statistics on it causes a crash.}
\label{fault2}
\end{figure}

\begin{figure}
{\scriptsize 
\begin{code}
shapefilelist0 = sorted(glob.glob(arcpy.env.workspace + "\\*.shp"))         \textcolor{black!60}{\# STEP 0}
shapefile0 = shapefilelist0 [0]                                            \textcolor{black!60}{\# STEP 1}
featureclass0 = shapefile0                                                 \textcolor{black!60}{\# STEP 2}
\textcolor{black!60}{\#[}
classorlayer0 = featureclass0                                              \textcolor{black!60}{\# STEP 3}
fieldtype0 = "DOUBLE"                                                      \textcolor{black!60}{\# STEP 4}
\textcolor{black!60}{\#  or fieldtype0 = "TEXT"}
\textcolor{black!60}{\#  or fieldtype0 = "FLOAT"}
\textcolor{black!60}{\#  or fieldtype0 = "SHORT"}
\textcolor{black!60}{\#  or fieldtype0 = "LONG"}
\textcolor{black!60}{\#  or fieldtype0 = "DATE"}
\textcolor{black!60}{\#  swaps with step 6}
fieldname0 = "newf1"                                                       \textcolor{black!60}{\# STEP 5}
\textcolor{black!60}{\#  or fieldname0 = "newf3"}
\textcolor{black!60}{\#  swaps with steps 6 8}
\textcolor{black!60}{\#] (steps in [] can be in any order)}
\textcolor{black!60}{\#[}
insertcursor0 = arcpy.InsertCursor(classorlayer0)                          \textcolor{black!60}{\# STEP 6}
\textcolor{black!60}{\#  swaps with steps 4 5}
arcpy.AddField\_management(featureclass0,fieldname0,fieldtype0); report()   \textcolor{black!60}{\# STEP 7}
\textcolor{black!60}{\#] (steps in [] can be in any order)}
fieldname0 = "newf2"                                                       \textcolor{black!60}{\# STEP 8}
\textcolor{black!60}{\#  or fieldname0 = "newf3"}
\textcolor{black!60}{\#  swaps with step 5}
arcpy.AddField\_management(featureclass0,fieldname0,fieldtype0); report()   \textcolor{black!60}{\# STEP 9}
insertcursor0 = arcpy.InsertCursor(classorlayer0)                          \textcolor{black!60}{\# STEP 10}
\end{code}
}
\caption{Creating two insert cursors on a layer, after adding two fields to the feature class underlying it, causes a crash.}
\label{fault3}
\end{figure}

\begin{figure}
{\scriptsize
\begin{code}
shapefilelist0 = sorted(glob.glob(arcpy.env.workspace + "\\*.shp"))         \textcolor{black!60}{\# STEP 0}
\textcolor{black!60}{\#[}
shapefile0 = shapefilelist0 [0]                                            \textcolor{black!60}{\# STEP 1}
newlayer0 = "l1"                                                           \textcolor{black!60}{\# STEP 2}
\textcolor{black!60}{\#  or newlayer0 = "l2" }
\textcolor{black!60}{\#  swaps with step 3}
\textcolor{black!60}{\#] (steps in [] can be in any order)}
\textcolor{black!60}{\#[}
featureclass0 = shapefile0                                                 \textcolor{black!60}{\# STEP 3}
\textcolor{black!60}{\#  swaps with step 2}
classorlayer0 = newlayer0                                                  \textcolor{black!60}{\# STEP 4}
\textcolor{black!60}{\#  or classorlayer0 = featureclass0 }
\textcolor{black!60}{\#  or (}
\textcolor{black!60}{\#      newlayer0 = "l3"  ;}
\textcolor{black!60}{\#      classorlayer0 = newlayer0 }
\textcolor{black!60}{\#     )}
fieldtype0 = "FLOAT"                                                       \textcolor{black!60}{\# STEP 5}
\textcolor{black!60}{\#  or fieldtype0 = "TEXT" }
\textcolor{black!60}{\#  or fieldtype0 = "DOUBLE" }
\textcolor{black!60}{\#  or fieldtype0 = "SHORT" }
\textcolor{black!60}{\#  or fieldtype0 = "LONG" }
\textcolor{black!60}{\#  or fieldtype0 = "DATE" }
fieldname0 = "newf1"                                                       \textcolor{black!60}{\# STEP 6}
\textcolor{black!60}{\#  or fieldname0 = "newf3" }
\textcolor{black!60}{\#  swaps with step 11}
op0 = ">"                                                                  \textcolor{black!60}{\# STEP 7}
\textcolor{black!60}{\#  or op0 = "<" }
\textcolor{black!60}{\#  or op0 = "<=" }
\textcolor{black!60}{\#  or op0 = ">=" }
\textcolor{black!60}{\#  or op0 = "=" }
val0 = "10"                                                                \textcolor{black!60}{\# STEP 8}
\textcolor{black!60}{\#  or val0 = "20" }
\textcolor{black!60}{\#  or val0 = "30" }
\textcolor{black!60}{\#  or val0 = "100" }
\textcolor{black!60}{\#  or val0 = "1000" }
\textcolor{black!60}{\#] (steps in [] can be in any order)}
\textcolor{black!60}{\#[}
whereclause0 = '"' + fieldname0 + '" ' + op0 + str(val0)                   \textcolor{black!60}{\# STEP 9}
arcpy.AddField\_management(featureclass0,fieldname0,fieldtype0); report()   \textcolor{black!60}{\# STEP 10}
\textcolor{black!60}{\#] (steps in [] can be in any order)}
\textcolor{black!60}{\#[}
fieldname0 = "newf2"                                                       \textcolor{black!60}{\# STEP 11}
\textcolor{black!60}{\#  or fieldname0 = "newf3" }
\textcolor{black!60}{\#  swaps with step 6}
arcpy.MakeFeatureLayer\_management(featureclass0,newlayer0,where\_clause=whereclause0);
   report()                                                                \textcolor{black!60}{\# STEP 12}
\textcolor{black!60}{\#] (steps in [] can be in any order)}
searchcursor0 = arcpy.SearchCursor(classorlayer0,whereclause0)             \textcolor{black!60}{\# STEP 13}
\textcolor{black!60}{\#  or searchcursor0 = arcpy.SearchCursor(classorlayer0) }
arcpy.AddField\_management(featureclass0,fieldname0,fieldtype0); report()   \textcolor{black!60}{\# STEP 14}
srow0 = searchcursor0.next()                                               \textcolor{black!60}{\# STEP 15}
\textcolor{black!60}{\#  or srow1 = searchcursor0.next() }
\textcolor{black!60}{\#  or srow2 = searchcursor0.next()}
\end{code}
}
\caption{Advancing a search cursor on a layer, after adding a field to
  the underlying feature class twice, causes a crash.}
\label{fault4}
\end{figure}

Thus far, our fault-finding process has focused on crashes.  Because
test cases that cause crashes stop the testing process, it is critical
to identify and avoid behavior that causes crashes before proceeding
to add further correctness properties.  Additionally, other than
hard-to-detect data corruption, crashes may be the most frustrating
faults for a developer to fix.  When ArcGIS or a standalone ArcPy
script causes a system crash, there is no readable error message, or
symptom in incorrect data to use in debugging the program.
Understanding system core dumps or analyzing the XML logs produced by
the ArcGIS engine is difficult for end users.  It would be ideal to
fix all ArcPy crashes by (at least) changing the library behavior to
issue an error message on invalid calls, but in the absence of bug
fixes, it is helpful to identify the root causes of various crashes
for users.  These test cases all involve calls that, as far as we
can tell, are not forbidden by ArcPy documentation.

Figures \ref{fault1}-\ref{fault4} show four test cases that result in
an ArcPy crash (the Python interpreter stops functioning, terminating
the test prematurely).  If executed in ArcGIS, these fragments will
crash ArcGIS.  Because these test cases are 1-minimal \cite{DD},
normalized, and generalized \cite{ICSTnorm}, we are able to describe in some detail
the general sequence of actions that produces each crash.  We have
discovered a few other crash faults; these are either not yet
well understood, or (we believe) equivalent in underlying cause
to these crashes.  The crashes discovered also serve as a basic proof
of concept that ArcPy test generation works and discovers
unanticipated interactions of API calls in the system.

\subsection{First Crash Fault}

ArcPy crashes when the feature class from which a layer is produced is
deleted, and the layer is used in a {\tt SelectLayer} call (this
version shows an attribute-based selection, but location selection
will cause the same problem): (Figure \ref{fault1}).  The underlying issue seems to be that
while operations on a deleted feature class properly notify a user the
feature class does not exist, ArcPy or ArcGIS does not track that
layers depending on that feature should also be deleted/invalidated
when the feature class is deleted.  Layers are not copies
of a feature class, but essentially new views of a feature class.
This means that when the underlying feature class is modified or
deleted, the view needs to be updated to reflect that change, and this
is not always correctly implemented.


\subsection{Second Crash Fault}

ArcPy crashes when asked to compute statistics (wth the {\tt FIRST} or
{\tt LAST} statistics types) over a field of a layer, when that field
has been deleted from the underlying feature class:  Figure
\ref{fault2}.  This is possibly related to the first crash fault:
ArcGIS again does not seem to properly propagate changes to an underlying
feature class to layers (which seem to be views) created on that feature class using a {\tt
  MakeFeatureLayer} call.



\subsection{Third and Fourth Crash Faults}

Creating an insert cursor on a feature class, before or after adding
one field to the feature class, then adding a second new field, then
creating a second insert cursor, causes ArcPy to crash:  Figure
\ref{fault3}.  A similar, but not obviously equivalent problem is
shown in Figure \ref{fault4}, where the combination and cursor type
are different, but the same issue of cursors interacting with feature
class or layer changes appears.  It seems likely that ArcPy should
simply add a requirement that feature classes or layers with active
cursors should not be modified at all, except by the active cursor.


%\section{Testing from Saved States}
\section{Future Work}
\label{future}

\subsection{More Complete Correctness Checks}
\label{futureoracle}

\subsection{Differential Testing within One Version}
\label{sec:reftest}

\subsection{Testing Multiprocessing}

\subsection{Testing in the Cloud}
\section{The ArcPy TSTL Test Harness}
\label{harness}

\begin{figure}
{\scriptsize
\begin{code}
@import shutil, os, glob, arcpy, exceptions, gc
@from arcpy import ExecuteError
<@
def cleanupFiles():
    gc.collect() \# Get rid of cursors
    for l in ["l1","l2","l3"]:
    	arcpy.Delete\_management(l)
    
    for f in glob.glob("C:\\Arctmp\\*"):
        try:
            shutil.rmtree(f)
        except:
            print "UNABLE TO REMOVE:",f
    for i in xrange(0,1000000): \# Find a workspace without a lock
        new\_workspace = "C:\\Arctmp\\workspace." + str(i)
        if not os.path.exists(new\_workspace):
            break             
    shutil.copytree("C:\\Arcbase",new\_workspace)
    arcpy.env.workspace = new\_workspace
    print sorted(glob.glob(arcpy.env.workspace + "\\*.shp")),
    print sorted(glob.glob(arcpy.env.workspace + "\\*.lyr")),
    print sorted(glob.glob(arcpy.env.workspace + "\\*.gdb"))


def fcsInGdb(gdb):
    old\_workspace = arcpy.env.workspace
    arcpy.env.workspace = gdb
    fcs = []
    for fds in arcpy.ListDatasets('','feature') + ['']:
        for fc in arcpy.ListFeatureClasses('','',fds):
            fcs.append(os.path.join(gdb,fds,fc))
    arcpy.env.workspace = old\_workspace
    return fcs

def report():
    print arcpy.GetMessages()
@>

pool: <shapefilelist> 2
pool: <shapefile> 2 CONST

pool: <gdbfilelist> 2
pool: <gdbfile> 2 CONST

pool: <gdbfeatureclasslist> 2

pool: <featureclass> 4 CONST
pool: <classorlayer> 4 CONST

pool: <classorlayerlist> 2

pool: <spatialref> 2

pool: <prjfilelist> 2
pool: <prjfile> 2 CONST

pool: <transformlist> 2
pool: <transform> 2 CONST

pool: <newlayer> 2 CONST

pool: <layerlist> 2
pool: <layer> 2

pool: <fieldname> 2 CONST
pool: <fieldtype> 2 CONST
pool: <fieldlist> 2
pool: <fieldnamelist> 2

pool: <stattype> 2 CONST
pool: <statfields> 2

pool: <dist> 2 CONST

pool: <sorttype> 2 CONST
pool: <spatialsort> 2 CONST

pool: <sort> 1
pool: <sortlist> 2

pool: <joinattributes> 2

pool: <overlaptype> 2 CONST
pool: <selectiontype> 2 CONST
pool: <op> 2 CONST
pool: <val> 2 CONST
pool: <whereclause> 2 CONST

pool: <errtable> 1 CONST
pool: <polytable> 1 CONST
pool: <stattable> 1 CONST

pool: <insertcursor> 3
pool: <searchcursor> 3
pool: <updatecursor> 3

pool: <irow> 3
pool: <srow> 3
pool: <urow> 3

init: cleanupFiles()

log: 1 arcpy.GetMessages()
\end{code}
}
\caption{ArcPy TSTL test harness definition preamble (pools, functions, logging).}
\label{preamble}
\end{figure}

\begin{figure}
{\scriptsize
\begin{code}
<gdbfilelist> := sorted(glob.glob(arcpy.env.workspace + "\\*.gdb"))
len(<gdbfilelist,1>) >= 1 -> <gdbfile> := <gdbfilelist> [0]
<gdbfilelist> = <gdbfilelist> [1:]

<shapefilelist> := sorted(glob.glob(arcpy.env.workspace + "\\*.shp"))
len(<shapefilelist,1>) >= 1 -> <shapefile> := <shapefilelist> [0]
<shapefilelist> = <shapefilelist> [1:]

<layerfile> := arcpy.env.workspace + <["\\new1.lyr", "\\new2.lyr", "\\new3.lyr"]>

<shapefile> := arcpy.env.workspace + <["\\new1.shp", "\\new2.shp", "\\new3.shp"]>

<prjfilelist> := sorted(glob.glob
   ("C:\\Program Files (x86)\\ArcGIS\\Desktop10.3\\Reference Systems\\*.prj"))
len(<prjfilelist,1>) >= 1 -> <prjfile> := <prjfilelist> [0]
<prjfilelist> = <prjfilelist> [1:]

<transformlist> := arcpy.ListTransformations(<spatialref>,<spatialref>)
<transformlist> = <transformlist> [1:]
len(<transformlist,1>) >= 1 -> <transform> := <transformlist> [0]

<newlayer> := <["l1", "l2", "l3"]>

<spatialref> := arcpy.SpatialReference(<prjfile>)

<gdbfeatureclasslist> := fcsInGdb(<gdbfile>)
<gdbfeatureclasslist> = <gdbfeatureclasslist>[1:]

<featureclass> := <shapefile>
len(<gdbfeatureclasslist,1>) >= 1 -> <featureclass> := <gdbfeatureclasslist>[0]

<classorlayer> := <featureclass>
<classorlayer> := <newlayer>

<classorlayerlist> := []
<classorlayerlist>.append(<classorlayer>)

<fieldtype> := <["TEXT", "FLOAT", "DOUBLE", "SHORT", "LONG", "DATE"]>

<fieldname> := <["newf1", "newf2", "newf3"]>

<dist> := <["100 Feet", "500 Feet", "1000 Feet", "1 Mile", "2 Miles">]

<joinattributes> := <["ALL", "NO\_FID", "ONLY\_FID"]>

<overlaptype> := <["INTERSECT", "CONTAINS", "COMPLETELY\_CONTAINS", "WITHIN",
   "SHARE\_A\_LINE\_SEGMENT\_WITH", "CROSSED\_BY\_THE\_OUTLINE\_OF"]>

<selectiontype> := <["NEW\_SELECTION", "ADD\_TO\_SELECTION", "REMOVE\_FROM\_SELECTION",
   "SUBSET\_SELECTION", "SWITCH\_SELECTION", "CLEAR\_SELECTION"]>

<sorttype> := <["ASCENDING", "DESCENDING">]

<spatialsort> := <["UR", "UL", "LR", "LL", "PEANO"]>

<sort> := [<fieldname>,<sorttype>]
<sortlist> := []
<sortlist>.append(<sort>)

\{IOError\} <insertcursor> := arcpy.InsertCursor(<classorlayer>)
\{IOError\} <insertcursor> := arcpy.InsertCursor(<classorlayer>,<spatialref>)
\{IOError\} <searchcursor> := arcpy.SearchCursor(<classorlayer>)
\{IOError,exceptions.RuntimeError\} <searchcursor> := arcpy.SearchCursor
   (<classorlayer>,<whereclause>)
\{IOError,exceptions.RuntimeError\} <searchcursor> := arcpy.SearchCursor
   (<classorlayer>,<whereclause>,<spatialref>)
\{IOError\} <updatecursor> := arcpy.UpdateCursor(<classorlayer>)
\{IOError\} <updatecursor> := arcpy.UpdateCursor(<classorlayer>,<spatialref>)

<irow> := <insertcursor>.newRow()
\{exceptions.RuntimeError\} <insertcursor>.insertRow(<irow>)

<irow> := <insertcursor>.next()
<urow> := <updatecursor>.next()
<srow> := <searchcursor>.next()

\{exceptions.RuntimeError\} <val> := <irow>.getValue(<fieldname>)
\{exceptions.RuntimeError\} <val> := <srow>.getValue(<fieldname>)
\{exceptions.RuntimeError\} <val> := <urow>.getValue(<fieldname>)

\{exceptions.RuntimeError\} <irow>.setValue(<fieldname>,<val>)

\{exceptions.RuntimeError\} <urow>.setNull(<fieldname>)
\{exceptions.RuntimeError\} <urow>.setValue(<fieldname>,<val>)
\{exceptions.RuntimeError\} <updatecursor>.deleteRow(<urow>)
\{exceptions.RuntimeError\} <updatecursor>.updateRow(<urow>)

<op> := <[">", "<", "<=", ">=", "=", "!=">]  

<val> := <["10", "20", "30", "100", "1000">]

<whereclause> := '"' + <fieldname> + '" ' + <op> + str(<val>)

<whereclause> := <whereclause> + ' AND ' + <whereclause>

<whereclause> := <whereclause> + ' OR ' +  <whereclause>

<whereclause> := 'NOT' + <whereclause>

<errtable> := arcpy.env.workspace + "\\geomerr.dbf"

<polytable> := arcpy.env.workspace + "\\polyneig.dbf"

<stattable> := arcpy.env.workspace + "\\stats.dbf"

\{IOError\} <fieldlist> := arcpy.ListFields(<classorlayer>)
len(<fieldlist,1>) >= 1 -> <fieldname> := <fieldlist> [0].name
<fieldlist> = <fieldlist> [1:]

<fieldnamelist> := []
<fieldnamelist>.append(<fieldname>)

<stattype> := <["SUM", "MEAN", "MIN", "MAX", "RANGE", "STD", "COUNT", "FIRST", "LAST"]>

<statfields> := []
<statfields>.append([<fieldname>,<stattype>])
\end{code}
}
\caption{ArcPy TSTL test harness actions, part 1.}
\label{actions1}
\end{figure}

\begin{figure}
{\scriptsize
\begin{code}
\{ExecuteError\} arcpy.MakeFeatureLayer\_management(<featureclass>,<newlayer>); report()

\{ExecuteError\} arcpy.MakeFeatureLayer\_management(<featureclass>,<newlayer>,
   where\_clause=<whereclause>); report()

\{ExecuteError\} arcpy.Project\_management(<featureclass>,<featureclass>,<spatialref>,
   <transform>); report()

\{ExecuteError\} arcpy.AddField\_management(<featureclass>,<fieldname>,<fieldtype>);
   report()

\{ExecuteError\} arcpy.DeleteField\_management(<featureclass>,<fieldname>); report()

\{ExecuteError\} arcpy.Buffer\_analysis(<classorlayer>,<featureclass>,<dist>); report()

\{ExecuteError\} arcpy.Buffer\_analysis(<classorlayer>,<featureclass>,<dist>,
   dissolve\_option="ALL"); report()

\{ExecuteError\} arcpy.Buffer\_analysis(<classorlayer>,<featureclass>,<dist>,
   dissolve\_option="LIST",dissolve\_field=<fieldnamelist>); report()

\{ExecuteError\} arcpy.Erase\_analysis(<classorlayer>,<classorlayer>,<featureclass>);
   report()

\{ExecuteError\} arcpy.Erase\_analysis(<classorlayer>,<classorlayer>,<featureclass>,
   cluster\_tolerance=<dist>); report()

\{ExecuteError\} arcpy.Intersect\_analysis(<classorlayerlist>,<featureclass>); report()

\{ExecuteError\} arcpy.Intersect\_analysis(<classorlayerlist>,<featureclass>,
   join\_attributes=<joinattributes>); report()

\{ExecuteError\} arcpy.Intersect\_analysis(<classorlayerlist>,<featureclass>,
   cluster\_tolerance=<dist>); report()

\{ExecuteError\} arcpy.Intersect\_analysis(<classorlayerlist>,<featureclass>,
   join\_attributes=<joinattributes>,cluster\_tolerance=<dist>); report()

\{ExecuteError\} arcpy.Union\_analysis(<classorlayerlist>,<featureclass>); report()

\{ExecuteError\} arcpy.Union\_analysis(<classorlayerlist>,<featureclass>,
   join\_attributes=<joinattributes>); report()

\{ExecuteError\} arcpy.Union\_analysis(<classorlayerlist>,<featureclass>,
   cluster\_tolerance=<dist>); report()

\{ExecuteError\} arcpy.Union\_analysis(<classorlayerlist>,<featureclass>,
   join\_attributes=<joinattributes>,cluster\_tolerance=<dist>); report()

\{ExecuteError\} arcpy.SpatialJoin\_analysis(<classorlayer>,<classorlayer>,
   <featureclass>); report()

\{ExecuteError\} arcpy.SymDiff\_analysis(<classorlayer>,<classorlayer>,<featureclass>);
   report()

\{ExecuteError\} arcpy.SymDiff\_analysis(<classorlayer>,<classorlayer>,<featureclass>,
   join\_attributes=<joinattributes>); report()

\{ExecuteError\} arcpy.SymDiff\_analysis(<classorlayer>,<classorlayer>,<featureclass>,
   join\_attributes=<joinattributes>,cluster\_tolerance=<dist>); report()

\{ExecuteError\} arcpy.PolygonNeighbors\_analysis(<classorlayer>,~<polytable>); report()

\{ExecuteError\} arcpy.Statistics\_analysis(<classorlayer>,~<stattable>,<statfields>);
   report()

\{ExecuteError\} arcpy.SelectLayerByLocation\_management(<newlayer>,
   select\_features=<newlayer>,overlap\_type=<overlaptype>)

\{ExecuteError\} arcpy.SelectLayerByLocation\_management(<newlayer>,
   select\_features=<newlayer>,overlap\_type=<overlaptype>,search\_distance=<dist>)

\{ExecuteError\} arcpy.SelectLayerByLocation\_management(<newlayer>,
   select\_features=<newlayer>,overlap\_type=<overlaptype>,search\_distance=<dist>,
   selection\_type=<selectiontype>)

\{ExecuteError\} arcpy.SelectLayerByAttribute\_management(<newlayer>,
   selection\_type=<selectiontype>,where\_clause=<whereclause>)

\{ExecuteError\} arcpy.Select\_analysis(<classorlayer>,<featureclass>,
   where\_clause=<whereclause>)

\{ExecuteError\} arcpy.CopyFeatures\_management(<featureclass>,<featureclass>); report()

\{ExecuteError\} arcpy.Sort\_management(<featureclass>,<featureclass>,<sortlist>);
   report()

\{ExecuteError\} arcpy.Sort\_management(<featureclass>,<featureclass>,<sortlist>,
   <spatialsort>); report()

\{ExecuteError\} arcpy.Sort\_management(<featureclass>,<featureclass>,
   [["Shape",<sorttype>]],<spatialsort>); report()

\{ExecuteError\} arcpy.CheckGeometry\_management(<classorlayer>,~<errtable>); report()

\{ExecuteError\} arcpy.CheckGeometry\_management(<classorlayerlist>,~<errtable>); report()

\{ExecuteError\} arcpy.Delete\_management(<featureclass>); report()
\end{code}
}
\caption{ArcPy TSTL test harness definition actions, part 2.}
\label{actions2}
\end{figure}



Figures \ref{preamble}-\ref{actions2} show a
version of the actual ArcPy test harness.  This version can reproduce the faults
described in this paper, though in practice some faults are easier to
detect than others, and for practical testing it is best to disable,
e.g., the various {\tt Delete} calls and to use guards to prevent all modifications of
layers or classes on which cursors are active.  Compiled to Python,
this harness defines nearly 2,000 actions, and the standalone
interface is nearly 60KLOC.  Using the harness involves first
compiling it to a Python class, then loading a test case generator
such as the random tester provided with TSTL into a Python environment
that has access to ArcPy.  For our experiments, we compiled the TSTL
using the Cygwin Python installation (for easy command-line access) but ran
tests in the IDLE environment installed with ArcGIS.
Running tests involves no complications beyond modifying parameters of
the random tester, if desired (setting a time limit for testing, the
length of test cases \cite{ASE08}, and whether to search for failures or produce
coverage regression tests, for example).  In most cases, the random tester
eventually terminates abnormally, and the test case causing the crash
is stored in a file in the ArcPy harness directory.  For regression
generation, the tool produces files of the same format to obtain
coverage of ArcPy code.

Figure \ref{preamble} contains the small amount of code needed to
prepare a temporary workspace for each test sequence.  This code
handles deleting any live cursors (the pool variables are guaranteed
to be deleted before each test, and garbage collection ensures the
cursors are deactivated), removing any layers created on feature
classes, and then setting the environment.  The system will scan for a
``free'' environment location, to handle cases where locks prevent
re-using an old directory\footnote{The problem of locked directories
  seldom surfaces, now that layers are deleted before each test, but
  may be needed when saving states.}.  The other utility functions
allow discovering the feature classes in a geodatabase and produce a
report on screen when a complex ArcGIS operation completes
successfully.

Figure \ref{actions1} shows the actions that create
input parameters for ArcGIS engine calls, primarily.  Many of these
simply pick some numeric or string constant (and the sets of constants
could be expanded, at the cost of more expensive normalization and
generalization).  Others, such as SQL query generation, are more
complex, with recursive expansion to theoretically unlimited query
length.  

Finally, Figure \ref{actions2} shows the TSTL for the actual
ArcGIS toolbox calls.  So far, this includes only a small subset of
the functions defined in the Management and Analysis toolboxes.  Note
that after each call there is a call to {\tt report}.  In pure random
testing, successful calls are infrequent enough that it is useful
for the user to see the ArcGIS messages produced by successful calls.
By turning on logging, the user can also see unsuccessful call
messages, but this tends to produce an overwhelming amount of output.

One critical design decision taken early in the development of this
harness was that there is no explicit definition of the starting data
used for testing.  Any data stored in the {\tt Arcbase} directory can
be used, and the harness supports both shapefiles and file geodatabases.
This has two purposes:  first, testing can be customized to use any
starting data, including a user's own shapefiles.  Second, this makes
it easy to implement deep state testing, since no assumptions are made
about the structure or number of data files used.  The lack of
dependence on specific files is implemented by having the test harness
use Python's {\tt glob} function to collect all files of a given
extension in a directory in a list, then chose an item from the list.
In order to make sure that behavior is deterministic (up to the choice
of actual data files), the glob results are sorted.  This approach
does bias random testing to using files earlier in alphabetical order,
but we do not expect base testing data to include a large number of
files.

The key driving requirements for this harness design are given in the
title of this paper:  the test harness must be \emph{extensible} and
it must be \emph{usable}.

Because ArcPy is large and complex, the effort to produce a
complete test harness and more effective specification of correctness
is a long-term effort, and may be carried out by other users:  the
harness therefore must be \emph{extensible}.  The
harness should also be suitable for use by ArcPy developers testing
how their own extensions to ArcPy interact with the base ArcPy API.
Adding new API calls to the harness should be easy:  we have defined
the pools for basic ArcPy object types and provided tools that should
handle much more complex test cases.  The decision discussed above to
make the test harness independent of specific data files is a good
example of our emphasis on extensibility.

The idea that other ArcPy developers, testing researchers, and perhaps
(end-user) software developers in other fields looking for a large
scale example of how to test systems with TSTL, will need to read and
modify the code also drives our emphasis on a truly usable system.
Usability has a more direct impact on the modifications to the TSTL
language and tools than on the test harness itself, but the test
harness is developed in the context of the improvements to the TSTL
ecosystem, such as standalone tests and normalization and
generalization.  However, some usability choices are decisions about
how to write the harness.  Consider the calls to {\tt
  SelectLayerByLocation\_management} shown in Figure \ref{actions2}.
The three lines of code could be more concisely expressed in one line using the {\tt <,[} construct:

\begin{code}
\{ExecuteError\} arcpy.SelectLayerByLocation\_management(
   <newlayer>,select\_features=<newlayer>,overlap\_type=
   <overlaptype><,[,search\_distance=<dist>,,],>
   <,[,selection\_type=<selectiontype>,,],>)
\end{code}

However, this is difficult to read, even for the TSTL developer, and
seems likely to discourage GIS developers trying to understand and
extend the harness.  There is a tension between, on the one hand, concise, abstracted
expression of all possible test actions and, on the other hand,
concrete readable connection between test actions and the lines of code that
appear in real ArcPy scripts.  We aim to stay closer to concrete
forms, even at some cost in increased length for the harness.  An
interesting observation is that this preference on the one hand makes
changing the harness more difficult --- to change the name of an API
call requires multiple edits if multiple parameter arities are used;
on the other hand, changing the code using {\tt <,[} requires
understanding the meaning of the code, each time, which resembles the
difficulty of altering a complex regular expression \cite{RegExp}.  In
this case, the best solution might be to add a specialized construct
to TSTL to handle optional elements of an action --- despite the fact
that the {\tt <,[} construct can express optional elements easily.
Even with such an option, it may be easier for developers to read and
understand the implications of individual calls, however, if they are
written out in the harness, rather than combinatorally generated by
the TSTL compiler.

In the long run, these issues are as complex as the questions of
abstraction vs. ease-of-understanding that have concerned designers
and users of programming languages since early in the history of
computer science.  As test definition comes closer to a specialized
kind of declarative programming, our understanding of the tradeoffs
will likely improve.
\section{Conclusions and Future Work}

This paper introduces test normalization and generalization.  The
methods presented are significant steps towards a difficult goal: providing
users of automated testing with a \emph{single test, as short and
  simple as possible, for each underlying fault in the SUT}, and
\emph{annotations describing the general conditions under which the
  fault manifests as failure}.  Normalization approaches this ideal by
rewriting numerous distinct failing tests into a smaller, often
minimal, set of simpler tests.  Generalization uses
automated experiments to distinguish essential and accidental elements of
a test.   In our experiments, normalization reduced the
number of failures to examine by well over an order of magnitude, often  to the ideal of one per fault.  The algorithms for normalization and generalization rely
on the ability of TSTL \cite{NFM15,ISSTA15} to define a total order
over test actions, based on an abstract form for tests, suitable
for term rewriting.  TSTL-based normalization is thus applicable to
any SUT, any source language, and many different test generation methods,
including those that already produce short tests
\cite{FA11,SoftBET}.  TSTL is currently available in a well-tested Python
version with full normalization and generalization implementations;
there is also a
beta version, with limited normalization, for Java.


 The goal of normalization and generalization can also be pursued in
settings other than API sequence or string grammar testing.  The
difficulties of defining a normal form for JavaScript \cite{jsfunfuzz}
or C \cite{csmith} tests are non-trivial, but not obviously
overwhelming \cite{CReduce}. Less effective methods
than ours might still aid debugging and assist fuzzer taming
\cite{PLDI13}.  Simple generalization (e.g., is this numeric constant
essential, can these two statements be swapped?) and a limited form of
fresh value generalization should be easy to apply, even for complex
programming language tests.  

TSTL is available in a working version for Python \cite{tstl} that
supports normalization and generalization.  Further experimental
evaluation of normalization and generalization over more SUTs is
important to quantify effectiveness and motivate new rewrites and
generalizations.  The TSTL implementations are designed to allow these
to be easily added, in order to bring testing closer to the
goal of ``one test to rule them all.''



% BibTeX users please use one of
%\bibliographystyle{spbasic}      % basic style, author-year citations
\bibliographystyle{spmpsci}      % mathematics and physical sciences
%\bibliographystyle{spphys}       % APS-like style for physics
\bibliography{bibliography}   % name your BibTeX data base

% Non-BibTeX users please use
%\begin{thebibliography}{}
%
% and use \bibitem to create references. Consult the Instructions
% for authors for reference list style.
%
%\bibitem{RefJ}
% Format for Journal Reference
%Author, Article title, Journal, Volume, page numbers (year)
% Format for books
%\bibitem{RefB}
%Author, Book title, page numbers. Publisher, place (year)
% etc
%\end{thebibliography}

\end{document}
% end of file template.tex


\section{Introduction}

\subsection{Esri ArcPy}


Esri (the Environmental Systems Research Institute) is the single
largest Geographic Information System (GIS) software vendor, with about 40\%
of global market share.  Esri's ArcGIS tools are extremely widely
used for GIS analysis, in government, scientific research, commercial
enterprises, and education.  Automation of complex GIS analysis and
data management is a frequent need, and Esri has long provided tools
for programming their GIS software tools.  The current method of
choice is a Python site-package, ArcPy \cite{ArcPy}.  ArcPy is a complex library,
with dozens of classes and hundreds of functions distributed over
a variety of of toolboxes.  Most of the code involved in ArcPy
functionality is C++ for which source is unavailable (the source for
the actual ArcGIS tools); the Python source alone is over 50,000 lines
of code.

\subsection{Automated Testing for the Rest of the World}

Previous work on automated testing for APIs has been largely carried
out by software testing researchers only, or (at most) by individuals
who are primarily software developers.  This paper presents work
largely directed by an individual (the first author) who is not a
software developer by profession or education, but a GIS analyst.  The
problem of end-user testing \cite{burnettEUSE,Silos,rothermelTOSEM} is
known to be difficult, and previous work has focused on end-users of
systems less like traditional programming, e.g. spreadsheets, visual
languages, or machine-learning systems.  This paper partly aims to
describe a case study in how a non-expert in testing who is familliar
with a software library can adapt existing tools to test that library.